\documentclass{article}
\usepackage{graphicx}
\usepackage{array}
\usepackage{hyperref}
\usepackage{fullpage}
\usepackage{listings}
\usepackage{enumitem}
\usepackage{lstautogobble}
\usepackage[T1]{fontenc}
\usepackage{subfiles}
\usepackage{nameref}
% This makes it so the web pages don't have indents as most of the time
% they are just annoying
\setlength{\parindent}{0pt}

\newcolumntype{L}{>{\centering\arraybackslash}m{2.5cm}}

\usepackage{textcomp}
\lstset{
  upquote=true,
  basicstyle = \ttfamily,
  columns=fullflexible
  escapeinside=||,
  autogobble
}

% Allows for assertions and clarifications.
\newcommand{\code}[1]{\lstinline{#1}}
\newcommand{\assertion}[2]{\textbf{Assertion: }#1 (\hyperlink{#2}{#2})}
\newcommand{\assertionref}[1]{\hyperlink{#1}{#1}}
\newcommand{\assertiondest}[1]{\hypertarget{#1}{\textbf{#1}:}}
\newcommand{\clarification}[2]{\textbf{Clarification: }#1 (\hyperlink{#2}{#2})}
\newcommand{\clarificationdest}[1]{\hypertarget{#1}{\textbf{#1}:}}

% --- This section allows for tex4ht only control statments
% From http://tex.stackexchange.fom/questions/93852/what-is-the-correct-way-to-check-for-latex-pdflatex-and-html-in-the-same-latex
\makeatletter
\edef\texforht{TT\noexpand\fi
  \@ifpackageloaded{tex4ht}
    {\noexpand\iftrue}
    {\noexpand\iffalse}}
\makeatother
% -----------------------------------------------------------------------------

\begin{document}
  \includegraphics[width=.5\textwidth]{static/GazpreaLogo.png}

% Put a tile and add some extra space into the pdf.
\ifpdf
  \LARGE
  \textbf{\textit{Gazprea} Spec}
  \normalsize
  \hspace{.7in}
\fi

% Change the depth of the front page ToC.
\ifdefined\HCode
  \Configure{tableofcontents*}{chapter,section,subsection}
\fi

\textit{Gazprea} is derived from a language originally designed at the IBM Hardware Acceleration
Laboratory in Markham, ON.

\section{Keywords}
\label{sec:keywords}
\subfile{sections/keywords.tex}

\section{Identifiers}
\label{sec:identifiers}
\subfile{sections/identifiers.tex}

\section{Comments}
\label{sec:comments}
\subfile{sections/comments.tex}


\section{Streams}
\label{sec:streams}
\subfile{sections/streams/streams.tex}

\section{Declarations}\label{sec:declaration}

  Variables must be declared before they are used. A variable may be declared anywhere within a \texttt{function}, or
  \texttt{procedure}, but it must not be referenced in the program before it has been declared.

  A variable may be declared in a couple different ways, and may have various type specifiers, such as \texttt{const},
  applied to them. Aside from a few special cases with \texttt{vectors, matrices,} and \texttt{tuples} declarations
  have the following formats:

  \begin{lstlisting}
    <specifiers> <type> <identifier> = <expression>;
    <specifiers> <type> <identifier>;
  \end{lstlisting}

  Both declarations are creating a variable with the name given by <identifier>\texttt{<identifier>} of type
  <type>\texttt{<type>}, and with specifiers given by <specifiers>\texttt{<specifiers>}.

  The first declaration explicitly initializes the value of the new variable with the value of
  <expression>\texttt{<expression>}.

  In \textit{Gazprea} all variables must be initialized in a well defined manner in order to ensure functional purity.
  If the variables were not initialized to a known value their initial value might change depending on when the
  program is run.  Therefore, the second declaration is equivalent to:

  \begin{lstlisting}
    <specifiers> <type> <identifier> = null;
  \end{lstlisting}

  For simplicity \textit{Gazprea} assumes that declarations can only appear at the beginning of a block. For instance
  this would not be legal in \textit{Gazprea}:

  \begin{lstlisting}
    integer i = 10;
    if (blah) {
      i = i + 1;
      real i = 0;  // Illegal placement of a declaration.
    }
  \end{lstlisting}

  because the declaration of the real version of \texttt{i} does not occur at the start of the block.

  The following declaration placement is lagal:

  \begin{lstlisting}
    integer i = 10;
    if (blah) {
      real i = 0;  // At the start of the block. All good.
      i = i + 1;
    }
  \end{lstlisting}

  The declaration of a variable happens after initialization. Thus it is illegal to refer to a variable within its own
  initialization statement.

  \begin{lstlisting}
    /* All of these declarations are illegal, they would result in garbage
       values. */
    integer i = i;
    integer v[10] = v[0] * 2;
  \end{lstlisting}

  An error message should be raised about the use of undeclared variables in these cases. If a variable of the same
  name is declared in an enclosing scope, then it is legal to use that in the initialization of a variable with the
  same name. For instance:

  \begin{lstlisting}
    integer x = 7;
    if (true) {
      integer y = x;  /* y gets a value of 7 */
      real x = x; /* Refers to the enclosing scope's 'x', so this is legal */

      /* Now 'x' refers to the real version, with a value of 7.0 */
    }
  \end{lstlisting}


\section{Type Qualifiers}
\label{sec:typeQualifiers}
\subfile{sections/typeQualifiers.tex}

\section{Types}
\label{sec:types}
\subfile{sections/types/types.tex}


\section{Type Inference}\label{sec:typeInference}


  In many cases the compiler can figure out what a variable's type, or a function's return type should be without an
  explicit type being provided. For instance, instead of writing:

  \begin{lstlisting}
    integer x = 2;
    const integer y = x * 2;
  \end{lstlisting}

  \textit{Gazprea} allows you to just write:

  \begin{lstlisting}
    var x = 2;
    const y = x * 2;
  \end{lstlisting}

  This is allowed because the compiler knows that the initialization expression, 2, has the type integer. Because of
  this the compiler can automatically give x an integer type. A \textit{Gazprea} programmer can use \texttt{var} or
  \texttt{const} for any declaration with an initial value expression, as long as the compiler can guess the type for
  the expression.

  One case where \texttt{var} or \texttt{const} will lead to an error is if the initial value is a polymorphic
  constant such as \texttt{null}, or \texttt{identity}.  \textit{Gazprea}'s type inference is simple, and only relies
  upon a single expression, so it can not discover the type of x in:

  \begin{lstlisting}
    var x = null;  /* Can't tell what type this is */
    const y = identity; /* Can't tell what type this is either */
    integer z = x;
  \end{lstlisting}

  Clearly the type that makes the most sense for x here would by \texttt{integer}, but \textit{Gazprea} only checks the
  initialization expression, and does not see how the variable x is used across statements. As a result an error
  should be raised in this situation.

  \textit{Gazprea} employs a very simple type inference algorithm on expressions.  It finds the type that makes sense
  across binary expressions in a bottom up fashion. For instance:

  \begin{lstlisting}
    /* The type for 'x' can be infered to be an integer. This expression
       can be rewritten as (null + 1) + null. The type inference algorithm
       checks (null + 1), and decides that since 1 is an integer 'null'
       must also be an integer because '+' can only be applied to numbers
       of the same type. Similarly it then infers that since (null + 1) is
       an integer (null + 1) + null is an integer as well, thus 'x' must
       be an integer. */

    var x = null + 1 + null;

    /* The simple type inference algorithm can not handle this case, and a
       type ambiguity error should be raised. Since this expression is
       (null + null) + 1 the type inference algorithm will try to figure
       out the type for (null + null), but since it doesn't know what
       either of the null values types are it can't. */

    var y = null + null + 1;
  \end{lstlisting}

\section{Vector/Matrix Type Checking}\label{sec:typeChecking}


  While the size of vectors and matrices may not always be known at compile time, there are instances where the
  compiler can perform length checks at compile time. For instance:

  \begin{lstlisting}
    integer v[3] = 1..10;
  \end{lstlisting}

  In cases like these the compiler is always able to catch the size mismatch, since the vector \texttt{1..10} is known
  at compile time.

  Your compiler should only handle the case where the initialization expression consists of an expression with only
  literal values (thus, it can be evaluated at compile time). Similarly the size of the declared vector must either be
  given with an expression of literal values, or not be provided. If a size mismatch is detected here the compiler
  should throw an error. The compiler should also be able to detect cases such as:

  \begin{lstlisting}
    integer vector v = 1;
  \end{lstlisting}

  where the length of the vector can not be determined at all.


\section{Type Casts}\label{sec:typeCasting}

  \textit{Gazprea} provides explicit type casting. A value may be converted to a
  different type using the following syntax:

  \begin{lstlisting}
    as<type>(value)
  \end{lstlisting}

  Where \texttt{value} is some expression, and \texttt{type} is the type we want to convert to. Conversions from one
  type to another may not be legal. For instance converting from an integer matrix to an integer does not make sense.

  This table summarizes all of the conversion rules between base types. The row is the type you are converting from,
  and the column is the type you are converting to.

  \begin{center}
    \begin{tabular}{|ll|LLLcL|}
    \cline{1-7}
    \multicolumn{2}{c}{} &
    \multicolumn{5}{c}{\textbf{To type}} \\
    \cline{3-7}
    \multicolumn{2}{c}{} &
    \multicolumn{1}{c}{boolean} &
    \multicolumn{1}{c}{integer} &
    \multicolumn{1}{c}{real} &
    \multicolumn{1}{c}{interval} &
    \multicolumn{1}{c}{character} \\
    \hline
    \multicolumn{1}{c}{} &
    boolean & id & 1 if true, 0 otherwise & 1.0 if true, 0.0 otherwise & N/A &
    \textit{ASCII} value 0 if false, 1 otherwise \\
    \multicolumn{1}{c}{} &
    integer & false if 0, true otherwise & id & real version of integer & N/A & unsigned integer value mod 256 \\
    \multicolumn{1}{c}{\textbf{From}} &
    real & N/A & truncate & id & N/A & N/A \\
    \multicolumn{1}{c}{} &
    interval & N/A & N/A & N/A & id & N/A \\
    \multicolumn{1}{c}{} &
    character & false if \textit{ASCII} 0, true otherwise & \textit{ASCII} value
    as integer & \textit{ASCII} value as real & N/A & id \\
    \hline
    \end{tabular}
  \end{center}

  In this table "id" means there is no change, "N/A" means the conversion is invalid and an error should be raised.

  Type casting can be done on vector and matrix types. The cast is applied to the base elements. Additionally the cast
  may be used to promote scalars, or truncate / pad vectors. For instance:

  \begin{lstlisting}
    real vector v[3] = [i in 1..3 | i + 0.3 * i];

    /* Convert the real vector to an integer vector */
    integer vector u[3] = as<integer vector>(v);

    /* Convert to integers and null pad */
    integer vector x[5] = as<integer[5]>(v);

    /* Truncate the vector */
    real vector y[2] = as<real vector[2]>(v);
  \end{lstlisting}

  An integer interval may be cast to an integer vector, or a real vector. If the vector is larger than the interval,
  then the vector is \texttt{null} padded.

  Type casting can be done between tuples as long as the tuples have the same number of fields. Each of the fields is
  cast as specified. For instance:

  \begin{lstlisting}
    tuple(integer, integer) int_tup = (1, 2);
    tuple(real, boolean) rb_tup = as<(real, boolean)>(int_tup);

    /* rb_tup has the value (1.0, true) now */
  \end{lstlisting}

  \texttt{var} and \texttt{const} are implicitly included in typecasts. Default is \texttt{var}


\section{Type Promotion}\label{sec:typePromotion}

  Automatic type promotion on scalar values only happens between integer values to real values. A real number can not
  be implicitly cast to an integer value, and doing so should raise a compiler error.

  Scalar values can also be promoted to vector, or matrix types. This occurs when a matrix, or vector is used in an
  operation with a scalar value of a base type that can be promoted to the base type of the matrix, or vector. When
  the scalar value is promoted it becomes a vector, or matrix, with identical dimensions, and every element of this
  new vector / matrix is equal to the original scalar value. An integer scalar may be promoted to a real vector, or a
  real matrix.

  Integer intervals may be promoted to integer vectors, and thus they may also be promoted to real vectors.

  Tuples may be promoted to tuples of another type if they have the same number of fields, and each field of the
  original can be automatically coerced into the type of the equivalent field in the other tuple.

  \begin{lstlisting}
    tuple(integer, integer) int_tup = (1, 2);
    tuple(real, real) real_tup = int_tup; /* real_tup == (1.0, 2.0) */
  \end{lstlisting}

  No other automatic conversions are legal, and a type error should be raised.  For instance:

  \begin{lstlisting}
    character x = 'a';
    integer y = 1 + x;
  \end{lstlisting}

  Should raise a type error. A character can be converted to an integer if it is explicitly cast, but it can not be
  done implicitly like this.


\section{Typedef}\label{sec:typedef}

  Custom names for types can be defined using \texttt{typedef}. Typedefs may only appear at global scope, they may not
  appear within functions or procedures. A typedef may use any valid identifier for the name of the type. After the
  typedef has been defined any global declaration or function defined may use the new name to refer to the old type.
  For instance:

  \begin{lstlisting}
    /* Is the same as declaring 'v' like this\ldots */
    const integer v[10] = [i in 1..10 | 7];

    typedef integer[2,3] two_by_three_matrix;

    two_by_three_matrix m = [i in 1..2, j in 1..3 | i + j];
  \end{lstlisting}


\section{Expressions}\label{sec:expressions}

  \subsection{Table of Operator precedence}\label{sec:operators}

    The following is a table containing all of the precedences and associativities of the operators in
    \textit{Gazprea}.

    \begin{center}
      \begin{tabular}{|r|l|l|}
        \hline
        \textbf{Precedence} &
        \multicolumn{1}{c}{\textbf{Operators}} &
        \textbf{Associativity} \\
        \hline
        (Highest) 1 & indexing              & left  \\
        2           & ..                    & N/A   \\
        3           & unary +, unary -, not & right \\
        4           & \textasciicircum      & right \\
        5           & *,/,\%                & left  \\
        6           & +,-                   & left  \\
        7           & by                    & left  \\
        8           & <,>,<=,>=             & left  \\
        9           & ==, !=                & left  \\
        10          & and                   & left  \\
        11          & or,xor                & left  \\
        (Lowest) 12 & ||                    & right \\
        \hline
      \end{tabular}
    \end{center}

  \subsection{Generators}\label{sec:generators}

    A generator may be used to construct either a vector or a matrix. A generator creates a value of a vector type
    when one domain variable is used, and a generator creates a value of a matrix type when two domain variables are
    used.  Any other number of domain variables will yield an error.

    A generator consists of either one or two domain expression. An additional expression is used on the right hand
    side in order to create the generated values. For example:

    \begin{lstlisting}
      integer vector v[10] = [i in 1..10 | i * i];
      /* v[i] == i * i */

      integer matrix M[2, 3] = [i in 1..2, j in 1..3 | i * j];
      /* M[i, j] == i * j */
    \end{lstlisting}

    The expression to the right of the bar "|", is used to generate the value at the given index, and must result in
    a value with the same type as the base type for the matrix or vector. Generators may be nested, and may be used
    within domain expressions. For instance, the generator below is perfectly legal:

    \begin{lstlisting}
      integer i = 7;

      /* The domain expression should use the previously defined i */
      integer vector v = [i in [i in 1..i | i] | [i in 1..10 | i * i][i]];

      /* v should contain the first 7 squares. */
    \end{lstlisting}


  \subsection{Filters}\label{sec:filters}

    Filters are used to accumulate elements into vectors. Each filter contains a single domain expression, and a
    list of predicates.

    The result of a filter operation is a tuple. This tuple contains a field for each of the predicates in order.
    Each field is a vector containing only the elements from the domain which satisfied the predicate expressions.
    Each filter result has an additional field which is a vector containing all of the values in the domain which
    did not satisfy any of the predicates. For example:

    \begin{lstlisting}
      /* x == ([3], [2], [2, 4], [1, 5]) */
      var x = [i in 1..5 & i == 3 or i == 2  or i \% 2 == 0];

      /* y == ([1, 3, 5], [2, 4]) */
      var y = [i in 1..5 & i \% 2 == 1);
    \end{lstlisting}

    There must be at least one predicate expression

  \subsection{Domain Expressions}
    Domain expressions can only appear within iterator loops, generators, and filters. A domain expression is a way
    of declaring a variable that is local to the loop, generator, or filter, that takes on values from intervals,
    and vectors in order.

    Domain expressions are essentially declarations, and so they follow the same scoping rules. For instance:

    \begin{lstlisting}
      integer i = 7;

      /* This will print 1234567 */
      loop i in 1..i {
        i -> out;
      }
    \end{lstlisting}

    Domain variables are not initialized when they are declared. For instance in loops they are initialized at the
    start of each execution of the loop's body statement. However, we may chain domain variables using commas, like
    in iterator loops, or matrix generators. Thus it is illegal to use a domain variable declared in the same chain
    of domain expressions, since the value may be uninitialized.

    \begin{lstlisting}
      integer i = 7;

      /* This is illegal because the i in "j in 1..i" refers to the domain
         variable i. An error should be raised in this case. */
      loop i in 1..i, j in 1..i {
         i * j -> out;
      }

      /* This is legal since i will be initialized whenever the inner loop
         is executed */
      loop i in 1..i {
        loop j in 1..i {
          i * j -> out;
        }
      }
    \end{lstlisting}

    The domain for the domain expression is only evaluated once. For instance:

    \begin{lstlisting}
      integer x = 1;

      /* 1..x is only evaluated the first time the loop executes, so it is
         simply 1..1, and not an infinite loop. */
      loop i in 1..x {
        x = x + 1;
      }
    \end{lstlisting}

    This is true for domain expressions within generators and filters as well.


\section{Statements}\label{sec:statements}

  \subsection{assignment Statements}\label{sec:assignment}

    In \textit{Gazprea} a variable may have different values throughout the execution of the program. Variables may
    have their values changed with an assignment statement. In the simplest case an assignment statement contains an
    identifier on the left hand side of an equals sign, and an expression with a compatible type on the right hand
    side.

    \begin{lstlisting}
      integer x = 7;

      x -> out;  /* Prints 7 */

      /* Give 'x' a new value */
      x = 2 * 3;  /* This is an assignment statement */

      x -> out;  /* Prints 6 */
    \end{lstlisting}

    Type checking must be performed on assignment statements. The expression on the right hand side must have a type
    that can be automatically promoted to the type of the variable. For instance:

    \begin{lstlisting}
      integer int_var = 7;
      real real_var = 0.0f;
      boolean bool_var = true;

      /* Since 'x' is an integer it can be promoted to a real number */
      real_var = int_var;  /* Legal */

      /* Real numbers can not be turned into boolean values automatically. */
      bool_var = real_var; /* Illegal */
    \end{lstlisting}

    Assignments can also be more complicated than this with vectors, matrices, and tuples. With matrices and vectors
    indices may be provided in order to change the value of a portion of the matrix or vector. For instance, with
    vectors:

    \begin{lstlisting}
      integer vector v = [0, 0, 0];

      /* Can assign an entire vector value -- change 'v' to [1, 2, 3] */
      v = [1, 2, 3];

      /* Change 'v' to [1, 0, 3] */
      v[2] = 0;

      /* Can also use vector indexing */
      v[[1, 3]] = [4, 5];  /* 'v' is now [4, 0, 5] */
    \end{lstlisting}

    Matrices can be treated similarly.

    \begin{lstlisting}
      integer matrix M = [[1, 1], [1, 1]];

      /* Change the entire matrix M to [[1, 2], [3, 4]] */
      M = [[1, 2], [3, 4]];

      /* Change a single position of M */
      M[1, 2] = 7;  /* M is now [[1, 7], [3, 4]] */

      /* Can use vector indexing on rows or columns.
         Uses all combinations of row / column coordinates */
    \end{lstlisting}


    Tuples also have a special unpacking syntax in \textit{Gazprea}. A tuple's field may be assigned to comma
    separated variables instead of a tuple variable. For instance:

    \begin{lstlisting}
      integer x = 0;
      real y = 0;
      real z = 0;

      tuple(integer, real) tup = (1, 2.0);

      /* x == 1, and y == 2.0 now */
      x, y = tup;

      /* Types can be promoted */

      /* z == 1.0, y == 2.0 */
      z, y = tup;

      /* Can swap: z == 2.0, y == 1.0 */
      z, y = (y, z);
    \end{lstlisting}

    The types of the variables must match the types of the tuple's fields, or the tuple's fields must be able to be
    automatically promoted to the variable's type. The number of variables in the comma separated list must match
    the number of fields in the tuple, if this is not the case an error should be raised.

    Assignments and initializations must perform a deep copy. It should not be possible to cause the aliasing of
    memory locations with an assignment. For instance:

    \begin{lstlisting}
      integer vector v = [1, 2, 3];
      integer vector w = v;

      w[2] = 0;  /* This must not affect 'v' */

      /* v has the value [1, 2, 3] */
      /* w has the value [1, 0, 3] */

      /* If you are not careful, you might copy the pointer of 'v' to 'w',
         which would cause them to be stored in the same location in memory. If
         this happens modifying 'w' would change 'v' as well.
       */
    \end{lstlisting}

    The above is a simple example using vectors. You must ensure that values can not be aliased with an assignment
    between any types, including vectors, matrices, and tuples.

    Variables may be declared as const, and in this case it is illegal for them to appear on the left hand side of
    an assignment expression. The compiler should raise on error when this is detected, since it does not make sense
    to change a constant value.

    The right hand side of an assignment statement is always evaluated before the left hand side. This is important
    for cases where procedures may change variables, for instance:

    \begin{lstlisting}
      v[x] = p(x);
      /* If p changes x then it is important that p(x) is executed before v[x] */
    \end{lstlisting}

  \subsection{Block Statements}\label{sec:block}

    A list of statements may be grouped into one statement using curly braces. This is called a block statement, and
    is similar to block statements in other languages such as \textit{C/C++}. As an example:

    \begin{lstlisting}
      {
        x = 3;
        z = 4;
        x -> out; "\n" -> out; z -> out; "\n" -> out;
      }
    \end{lstlisting}

    Is a block statement. Declarations can only appear at the start of a block.  Each block statement introduces a
    new scope that new variables may be declared in. For instance this is perfectly valid:

    \begin{lstlisting}
      int x = 3;
      int y = 0;
      real z = 0;

      {
        real x = 7.1;
        z = x;
      }

      y = x;
    \end{lstlisting}

    After execution this \texttt{y = 3} and \texttt{z = 7.1}.


  \subsection{If/Else Statements}\label{sec:conditional}

    An if statement takes a boolean value as a conditional expression, and a statement for the body. If the
    conditional expression evaluates to true, then the body is executed. If the conditional expression evaluates to
    false then the body of the if statement is not executed. If statements in \textit{Gazprea} do not require the
    conditional expression to be enclosed in parenthesis.

    \begin{lstlisting}
      integer x = 0;
      integer y = 0;

      /* Compute some value for x */

      if (x == 3) {
         y = 7;
      }

      /* At this point y will only be 7 if x == 3, and otherwise y will be
         0, assuming it did not change throughout the rest of the program.
       */
    \end{lstlisting}

    If statements are often paired with block statements, like in the above example. The if statement above could
    also be written as:

    \begin{lstlisting}
      if x == 3
        y = 7;
    \end{lstlisting}

    Since \texttt{y = 7;} is a statement it can be used as the body statement. All statements after this point are
    not in the body of the if statement. For instance:

    \begin{lstlisting}
      if x == 3
        y = 7;
        z = 32;
    \end{lstlisting}

    is actually equivalent to the following:

    \begin{lstlisting}
      if (x == 4) {
        y = 7;
      }

      z = 32;
    \end{lstlisting}

    \textit{Gazprea} is not sensitive to whitespace, so we could even write something like:

    \begin{lstlisting}
      if x == 3 y = 7;
    \end{lstlisting}

    An if statement may also be followed by an else statement. The else has a body statement just like the if
    statement, but this is only run if the conditional expression on the if statement fails.

    \begin{lstlisting}
      if x == 3
        y = 7;
      else
        y = 32;
    \end{lstlisting}

    Now if \texttt{x} does not have a value of 3, \texttt{y} is assigned a value of 32. This can be paired with if
    statements as well.

    \begin{lstlisting}
      y = 0;

      if (x < 0) {
        y = -1;
      }
      else if (x > 0) {
        y = 1;
      }

      /* y is negative if x is negative, positive if x is positive,
        and 0 if x is 0. */
    \end{lstlisting}


  \subsection{Loop}\label{sec:loop}
    \subsubsection{Infinite Loop}\label{sec:infLoop}
      \textit{Gazprea} provides an infinite loop, which continuously executes the body statement given to it. For
      instance:

      \begin{lstlisting}
        loop "hello!\n" -> out;
      \end{lstlisting}

      Would print "hello!" indefinitely. This is often used with block statements.

      \begin{lstlisting}
        /* Infinite counter */
        integer n = 0;

        loop {
          n -> out; "\n" -> out;
          n = n + 1;
        }
      \end{lstlisting}

    \subsubsection{Predicated Loop}\label{sec:predicatedLoop}

      A loop may also be provided with a control expression. The control expression automatically breaks from the
      loop if it evaluates to false when it is checked.

      The loop can be pre-predicated, which means that the control expression is tested before the body statement
      is executed. This is the same behaviour as while loops in most languages, and is written using the while
      token after the loop, followed by a boolean expression for the predicate. For example:

      \begin{lstlisting}
        integer x = 0;

        /* Print 1 to 10 */
        loop while x < 10 {
          x = x + 1;
          x -> out; "\n" -> out;
        }
      \end{lstlisting}

      A post-predicated loop is also available. In this case the control expression is tested after the body
      statement is executed. This also uses the while token followed by the control expression, but it appears at
      the end of the loop. Post Predicated loop statements must end in a semicolon.

      \begin{lstlisting}
        integer x = 10;

        /* Since the conditional is tested after the execution '10' is printed */
        loop x -> out; while x == 0;
      \end{lstlisting}


    \subsubsection{Iterator Loop}\label{sec:iteratorLoop}

      Loops can be used to iterate over the elements of an integer interval, or a vector of any type. This is done
      by using domain expressions (for instance i in v) in conjunction with a loop statement.

      When the domain is given by a vector, each time the loop is executed the next element of the vector is
      assigned to the domain variable. The elements of the domain vector are assigned to the domain variable
      starting from index 1, and going up to the final element of the vector. When all of the elements of the
      domain vector have been used the loop automatically exits. For instance:

      \begin{lstlisting}
        /* This will print 123 */
        loop i in [1, 2, 3] {
          i -> out;
        }
      \end{lstlisting}

      Integer intervals can also be used instead. In this case it is the same as iterating over a vector created
      from the interval using by 1. For instance, the above iterator loop is equivalent to the following:

      \begin{lstlisting}
        /* This will print 123 */
        loop i in 1..3 {
          i -> out;
        }
      \end{lstlisting}

      The domain is evaluated once during the first iteration of the loop. For instance:

      \begin{lstlisting}
        integer vector v = [i in 1..3 | i];

        /* Since the domain 'v' is only evaluated once this loop prints 1, 2,
           and then 3 even though after the first iteration 'v' is the zero
           vector. */
        loop i in v {
          v = 0;
          i -> out; "\n" -> out;
        }
      \end{lstlisting}

      Multiple domain expressions may be used by separating them with commas.

      \begin{lstlisting}
        loop i in u, j in v {
          "Hello!\n" -> out;
        }

        /* The above loop is equivalent to the loop below */

        loop i in u {
          loop j in v {
            "Hello!\n" -> out;
          }
        }
      \end{lstlisting}

      This can be done with as many domain expressions as desired.


  \subsection{Break}\label{sec:break}

    A break statement may only appear within the body of a loop. When a break statement is executed the loop is
    exited, and \textit{Gazprea} continues to execute after the loop. This only exits the innermost loop, which
    actually contains the break.

    \begin{lstlisting}
      /* Prints a 3x3 square of *'s */
      integer x = 0;
      integer y = 0;

      loop while y < 3 {
        y = y + 1;

        /* Normally this would loop forever, but the break exits this inner loop */
        loop {
          if x >= 3 break;

          x = x + 1;
          "*" -> out;
        }

        "\n" -> out;
      }
    \end{lstlisting}

    If a break statement is not contained within a loop an error must be raised.

  \subsection{Continue}\label{sec:continue}


    Similarly to \texttt{break}, \texttt{continue} may only appear within the body of a loop. When a
    \texttt{continue} statement is executed the innermost loop that contains the \texttt{continue} statements starts
    its next iteration.  \texttt{continue} stops the execution of the loop's body statement, the loop then continues
    as though the body statement finished its execution normally.

    \begin{lstlisting}
      /* Prints every number between 1 and 10, except for 7 */
      integer x = 0;

      loop while x < 10 {
        x = x + 1;

        if x == 7 continue;  /* Start at the beginning of the loop, skip 7 */

        x -> out; "\n" -> out;
      }
    \end{lstlisting}


  \subsection{Return}\label{sec:return}

    The return statement is used to stop the execution of a function or procedure.  When a function / procedure
    returns execution continues where the function was called. The return statement must be given a value that is
    compatible with the return type of the function / procedure, this will be the result of the function / procedure
    call. Here is an example:

    \begin{lstlisting}
      function square(integer x) returns integer {
        return x * x;
      }
    \begin{lstlisting}

    If the procedure has no returns clause, then it has no return type. In this case return is used as follows:

    \begin{lstlisting}
      procedure do_nothing() {
        return;
      }
    \end{lstlisting}


  \subsection{Stream Statements}

    Stream statements are the statements used to read and write values in \textit{Gazprea}.

    Output example:

    \begin{lstlisting}
      2 * 3 -> out;  /* Prints 6 */
    \end{lstlisting}

    Input example:

    \begin{lstlisting}
      integer x = null;
      x <- inp; /* Read an integer into x */
    \end{lstlisting}


\section{Functions}\label{sec:function}

  A function in \textit{Gazprea} has several requirements:

  \begin{itemize}
    \item All of the arguments are implicitly \texttt{const}, and can not be mutable.
    \item Functions can not perform any I/O.
    \item Functions can not rely upon any mutable state outside of the function.
    \item Functions can not call any procedures.
  \end{itemize}

  The reason for this is to ensure that functions in \textit{Gazprea} behave as pure functions. Every time you call a
  function with the same arguments it will perform the exact same operations. This has a lot of benefits. It makes
  code easier to understand if functions only depend upon their inputs and not some hidden state, and it also allows
  the compiler to make more assumptions and as a result perform more optimizations.

  \subsection{Syntax}

    A function is declared using the function keyword. Each function is given an identifier, and an arguments list
    enclosed in parenthesis. If no arguments are provided an empty set of parenthesis, \texttt{()}, must be used.
    The return type of the function is specified after the arguments using returns.

    A function can be given by a single expression. For instance:

    \begin{lstlisting}
      function times_two(integer x) returns integer = 2 * x;
    \end{lstlisting}

    This defines a function called times\_two which can be used as follows:

    \begin{lstlisting}
      /* Prints 8. value gets assigned the result of calling times_two with an
         argument of 4
       */
      integer value = times_two(4);

      value -> out; "\n" -> out;
    \end{lstlisting}

    Functions can have an arbitrary number of arguments. Here are some examples of functions with different numbers
    of arguments:

    \begin{lstlisting}
      /* A function with no arguments */
      function f() returns integer = 1;

      /* A function with two arguments */
      function pythag(real a, real b) returns real = (a^2 + b^2)^(1/2);

      /* A function with different types of arguments */
      function get(real vector a, integer i) returns real = a[i];
    \end{lstlisting}

    These can be called as follows:

    \begin{lstlisting}
      integer x = f(); /* x == 1 */
      real c = pythag(3, 4); /* Type promotion to real arguments. c == 5.0 */
      real value = get([i in 1..10 | i], 3); /* value == 3 */
    \end{lstlisting}

    A function's body can also be given by a block statement instead of a single expression. In this case the return
    value of the function is given with the return statement. A return statement must be reached by all possible
    control flows in the function before the end of the function is encountered.

    \begin{lstlisting}
      /* Invalid -- should cause a compiler error */
      function f (boolean b) returns integer {
        if (b) {
          return 3;
        }
      }

      /* Valid, all possible branches hit a return statement with a valid type */
      function g (boolean b) returns integer {
        if (b) {
          return 3;
        }
        else {
          return 8;
        }
      }
    \end{lstlisting}

    \texttt{f} is invalid since if \texttt{b == false}, then we reach the end of the function without a return
    statement, so we don't know what value \texttt{f(false)} should take on.

    \begin{lstlisting}
      /* This is invalid because if the loop ever finished executing the
         function would end before a return statement is encountered. In
         general the compiler can not tell when a loop would execute
         forever, so we make the assumption that all branches in the control
         flow could be followed. */
      function f() returns integer {
        integer x = 0;
        loop {
          x = x + 1;
        }
      }

      /* This is valid. Even though the loop goes on forever so that a
         return is never reached, execution never hits the end of the
         function without a return. */
      function g() return integer {
        integer x = 0;
        loop {
          x = x + 1;
        }

        return x;
      }
    \end{lstlisting}

    Each function has its own scope, but globals can be accessed within the function if they were declared before
    the function was defined.

  \subsection{Forward Declaration}

    Functions can be declared before they are defined in a \textit{Gazprea} file.  This allows function definitions
    to be moved to more convenient locations in the file.

    \begin{lstlisting}
      /* Forward declaration, no body */
      function f(integer x) returns integer;

      procedure main() returns integer {
        integer y = f(13);
        /* Can use this in main, even though the definition is below */
        return 0;
      }

      function f(integer x) returns integer = x^2;
    \end{lstlisting}


    If the type signatures of the forward declaration of the function and the definition of the function differ then
    an error must be raised. A function may only be declared once.

    A function that has a forward declaration must have a definition somewhere within the file. If the function does
    not have a definition then an error should be raised.


\section{Procedures}\label{sec:procedure}

  A procedure in \textit{Gazprea} is like a function, except that it does not have to be pure and as a result it may:

  \begin{itemize}
    \item Have arguments marked with var which can be mutated. By default arguments are \texttt{const} just like
    functions.
    \item A procedure may only accept a literal or expression as an argument if and only if the procedure declares
    that argument as \texttt{const}
    \item Procedures may perform I/O.
    \item A procedure can call other procedures.
    \item Procedures can only be called in assignment statements / procedure call statements.
    \item When used in an assignment statement the procedure may only be used with unary operations.
  \end{itemize}

  Aside from this (and the different syntax necessary to declare / define them), procedures are very similar to
  functions. The extra capabilities that procedures have makes them harder to reason about, test, and optimize.


  \subsection{Syntax}

    Procedures are almost exactly the same as functions, however the returns clause is optional. Since procedures
    can cause side effects, it makes sense to have procedures without a return value. For instance:

    \begin{lstlisting}
      procedure change_first(var integer vector v) {
        v[1] = 7;
      }

      procedure increment(var integer x) {
        x = x + 1;
      }
    \end{lstlisting}

    These procedures can be called as follows:

    \begin{lstlisting}
      integer x = 13;
      integer vector v[5] = 13;

      call change_first(v); /* v == [7, 13, 13, 13, 13] */
      call increment(x); /* x == 14 */
    \end{lstlisting}

    It is only possible to call procedures in this way. Functions must appear in expressions because they can not
    cause side effects, so using a function as a statement would not do anything, and thus \textit{Gazprea} should
    raise an error. If the procedure has a return value and is called in this fashion the return value is discarded.

    Procedures may also be called in expressions just like functions, but with a few more limitations. A procedure
    may never be called within a function, doing so would allow for impure functions. Procedures may only be called
    within assignment statements (procedures may not be used as the control expression in control flow expressions,
    for instance). The return value from a procedure call can only be manipulated with unary operators. It is
    illegal to use the results from a procedure call with binary expressions, for instance:

    \begin{lstlisting}
      /* p is some procedure with no arguments */
      var x = p(); /* Legal */
      var y = -p(); /* Legal, depending on the return type of p */
      var z = not p(); /* Legal, depending on the return type of p */
      var u = p() + p(); /* Illegal */
    \end{lstlisting}

    This restriction is made by \texttt{Gazprea} in order to allow for more optimizations.

    As long as they have an appropriate return type. The difference is that functions can be called within other
    functions, but procedures can not be used within functions since procedures may be impure. Procedures may only
    be called within procedures.


\subsection{Forward Declaration}

  Procedures can use forward declaration just like functions.


  \subsection{Main}

    Execution of a \textit{Gazprea} program starts with a procedure called \texttt{main}. This procedure takes no
    arguments, and has an integer return type. If a program is missing a main procedure an error should be raised.

    \begin{lstlisting}
      /* must be writen like this */
      procedure main() returns integer {
        var out = std_output();
        integer x = 1;
        x = x + x;
        x -> out;

        /* must have a return */
        return 0;
      }
    \end{lstlisting}

  \subsection{Aliasing}

    Since procedures can have mutable arguments, it would be possible to cause
    \href{http://en.wikipedia.org/wiki/Aliasing_(computing)}{aliasing}. In \textit{Gazprea} aliasing of mutable
    variables is illegal (the only case where any aliasing is allowed is that tuple members can be accessed by name,
    or by number, but this is easily spotted). This helps \textit{Gazprea} compilers perform more optimizations.
    However, the compiler must be able to catch cases where mutable memory locations are aliased, and an error
    should be raised when this is detected. For instance:

    \begin{lstlisting}
      procedure p(var integer a, var integer b, const integer c, const integer d) {
         /* Some code here */
      }

      procedure main() returns integer {
        integer x = 0;
        integer y = 0;
        integer z = 0;

        /* Illegal */
        call p(x, x, x, x); /* Aliasing, this is an error. */
        call p(x, x, y, y); /* Still aliasing, error. */
        call p(x, y, x, x); /* Argument a is mutable and aliased with c and d. */

        /* Legal */
          call p(x, y, z, z);
          /* Even though 'z' is aliased with 'c' and 'd' they are
          both const. */

        return 0;
      }
    \end{lstlisting}

    Whenever a procedure has a mutable argument x it must be checked that none of the other arguments given to the
    procedure are x. This is simple for scalar values, but more complicated when variable vectors and matrices are
    passed to procedures. For instance:

    \begin{lstlisting}
      call p(v[1..5], v[6..10]);
      /* p is some procedure with two variable vector arguments */
    \end{lstlisting}

    In this case the arguments technically wouldn't be aliased, since the vector slices represent different
    locations in memory, but since the vector slices may depend upon variables:

    \begin{lstlisting}
      call p(v[x], v[y]);
      /* p is some procedure with two variable vector arguments */
    \end{lstlisting}

    It is impossible to tell whether or not these overlap at compile time due to the halting problem. Thus for
    simplicity, whenever a vector or a matrix is passed to a procedure \textit{Gazprea} detects aliasing whenever the
    same vector / matrix is used, regardless of whether or not the sections used would overlap.  Thus, this should
    cause an error to be raised:

    \begin{lstlisting}
      call p(v[1..5], v[6..10]);
      /* p is some procedure with two variable vector arguments */
    \end{lstlisting}


\section{Globals}\label{sec:global}
  In \textit{Gazprea} values can be assigned to a global identifier. All globals must be declared \texttt{const}. If a
  global identifier is not declared with the \texttt{const} specifier, then an error should be raised. This
  restriction is in place since mutable global variables would ruin functional purity. If functions have access to
  mutable global state then we can not guarantee their purity.

  Globals must be initialized, but the initialization expressions must not contain any function calls, or procedures.
  If a global is initialized with an expression containing a function call, or a procedure call, then an error should
  be raised. Initializations of globals may refer to previously defined globals.


\section{Built In Functions}\label{sec:builtIn}

  \textit{Gazprea} has some built in functions. These built in functions may have some special behaviour that normal
  functions can not have, for instance many of them will work on vectors of any base type. Normally a function must
  specify the base type of a vector specified.

  The name of built in functions are reserved and a user program cannot define a function or a procedure with the same
  name as a built in function. If a declaration or a definition with the same name as a built-in function is
  encountered in a \textit{Gazprea} program, then the compiler should issue an error message.

  \subsection{Length}\label{sec:length}

    \texttt{length} takes a vector of any base type, and returns an integer representing the length of the vector.

    \begin{lstlisting}
      integer vector v = 1..5;

      length(v) -> out; /* Prints 5 */
    \end{lstlisting}


  \subsection{Rows and Columns}\label{sec:rowsColumns}

    The built-ins \texttt{rows} and \texttt{columns} operate on matrices of any dimension and type. \texttt{rows}
    returns the number of rows in a matrix, and \texttt{columns} returns the number of columns in the matrix.

    \begin{lstlisting}
      integer matrix M = [[1, 2, 3], [4, 5, 6]];

      rows(M) -> out; /* Prints 2 */
      columns(M) -> out; /* Prints 3 */
    \end{lstlisting}


  \subsection{Reverse}\label{sec:reverse}

    The reverse built-in takes any vector, and returns a reversed version of the vector.

    \begin{lstlisting}
      integer vector v = 1..5;
      integer vector w = reverse(v);

      v -> out; /* Prints 12345 */
      w -> out; /* Prints 54321 */
    \end{lstlisting}

  \subsection{Stream State}\label{sec:streamState}
    When reading in values from stdin of certain types it is possible that an error is encountered, or that the end
    of the stream has been encountered. In order to handle these situations the language provides a built in
    function:

    \begin{lstlisting}
      stream_state(inp)
    \end{lstlisting}

    This function can only be called with the input stream as a parameter, but it's general enough that it could be
    used if the language were expanded to include multiple input streams.

    When called \texttt{stream\_state} will return an integer value. The return value is an error code defined as
    follows:

    \begin{itemize}
      \item 0: Last read from the stream was successful
      \item 1: Last read from the stream encountered an error.
      \item 2: Last read from the stream encountered the end of the stream.
    \end{itemize}

    When an error is encountered the value assigned to the variable used in the stream read operation is
    \texttt{null}. When an end of stream is encountered \texttt{null} is assigned to variables of all types except
    for character. In this case character is assigned the ASCII EOF value.

    When a read from the stream is not successful, and \texttt{stream\_state} returns non-0, then the stream must be
    rewound to where it was before the read started.

    Reading a character never causes an error. The character will either be successfully read, or the end of the
    stream has been reached and character is assigned EOF for each subsequent read.

    When reading anything else if something is read which does not match the data type (for instance if the letter
    'a' is read while trying to parse an integer), then this is an error. In this case the value returned by the
    stream is \texttt{null}, and \texttt{stream\_state} should return 1. Or, if only whitespace and the end of the
    stream is encountered the stream is rewound to where it was before the read, \texttt{null} is returned, and
    \texttt{stream\_state} will return 2.


\section{Backend}

  You don't need to implement an interpreter for Gazprea. You only need to implement a llvm code generator.

  \subsection{Memory Management}\label{sec:memory}
     It is important that you are able to automatically free and allocate memory for vectors and matrices when they enter and exit scope.
     You may use \texttt{malloc} and \texttt{free} for these purposes. This will have to be done within the LLVM IR, as your runtime will
     not be able to tell when your variables should be freed.

     Below is an example of how to use \texttt{malloc} and \texttt{free} within LLVM:

     \begin{lstlisting}
      define i32 @main(i32 %argc, i8** %argv) {
        %1 = call i8* @malloc(i64 128)
        call void @free(i8* %1)
        ret i32 0
      }

      declare i8* @malloc(i64)
      declare void @free(i8*)
    \end{lstlisting}
    It is important that the code generated by your compiler has no memory leaks, and that all memory is freed as soon as possible.

  \subsection{Runtime Libraries} % TODO FINISH THIS SECTION

    If you make a runtime library, the runtime library must be implemented in a runtime directory (lib). Beware that in C++ there is additional name
    mangling that occurs to allow class functions. Thus, we recommend that all runtime functions should be written in C and not in C++. There is a
    Makefile in the (lib) folder designed to turn all \texttt{*.c} and \texttt{*.h} pairs into part of the unified runtime library
    \texttt{libruntime.a}. An example of how to make a runtime function is provided bellow.

    \texttt{functions.c}
    \begin{lstlisting}
    #include "functions.h"

    uint64_t factorial(uint64_t n) {
        uint64_t fact = 1;

        while (n > 0) {
            fact *= n;
            n--;
        }

        return fact;
    }
    \end{lstlisting}

    \texttt{functions.h}
    \begin{lstlisting}
    #pragma once

    #include <stdint.h>

    uint64_t factorial(uint64_t n);
    \end{lstlisting}

    If ou compiler was compiling the following input
    \texttt{Input file}
    \begin{lstlisting}
    3! + (2 + 7)!
    \end{lstlisting}
    Here is how to call the function in LLVM code.

    \texttt{LLVM src}
    \begin{lstlisting}
    target triple = "x86_64-pc-linux-gnu"
    define i32 @main() {

        ; Calls factorial on 3 for the first part of expression
        %0 = call i64 @factorial(i64 3)

        ; Adds 2 and 7 together
        %1 = add i64 2, 7

        ; Calls factorial of (2 + 7)
        %2 = call i64 @factorial(i64 %1)

        ; Adds the results of 3! and (2 + 7)!
        %3 = add i64 %0, %2

        ; Done, return 0
        ret i32 0
    }

    ; This makes the function available for calling
    declare i64 @factorial(i64)
    \end{lstlisting}

\section{Compiler Implementation --- Part 1}\label{sec:part1}
  This section lists the portions of the  \textit{Gazprea} specification that must be implemented to complete the part 1 of the compiler implementation.
All developers are advised to read the full
  specification for the language prior to start the implementation of Part 1 because decisions made while implementing Part 1 can make the implementation of Part 2 significantly more challenging. Thus, planning ahead for Part 2 is the recommended strategy.

  \begin{enumerate}[label*=\arabic*]
    \item \hyperref[sec:comments]{Comments}
    \item \hyperref[sec:types]{Types}
    \begin{enumerate}[label*=.\arabic*]
      \item \hyperref[sec:types]{Data Types}
      \begin{enumerate}[label*=.\arabic*]
        \item \hyperref[ssec:boolean]{Booleans}
        \item \hyperref[ssec:character]{Characters}
        \item \hyperref[ssec:integer]{Integers}
        \item \hyperref[ssec:real]{Reals}
        \item \hyperref[ssec:tuple]{Tuples}
      \end{enumerate}
      \item \hyperref[sec:typeQualifiers]{Type Qualifiers}
      \begin{enumerate}[label*=.\arabic*]
        \item var
        \item const
      \end{enumerate}
      \item \hyperref[sec:typeChecking]{Type Checking}
      \item \hyperref[sec:typePromotion]{Type Promotion}
      \item \hyperref[sec:typeCasting]{Type Casting}
      \item \hyperref[sec:typeInference]{Type Inference}
      \item \hyperref[sec:typedef]{Typedef}
    \end{enumerate}
    \item \hyperref[sec:statements]{Statements}
    \begin{enumerate}[label*=.\arabic*]
      \item \hyperref[sec:assignment]{Assignment Statement}
      \item \hyperref[sec:declaration]{Declaration Statement}
      \item \hyperref[sec:global]{Global values}
      \item \hyperref[sec:block]{Block Statement}
      \item \hyperref[sec:loop]{loop}
      \begin{enumerate}[label*=.\arabic*]
        \item \hyperref[sec:break]{Break}
        \item \hyperref[sec:continue]{Continue}
      \end{enumerate}
      \item \hyperref[sec:conditional]{Conditionals}
      \begin{enumerate}[label*=.\arabic*]
        \item if
        \item else
      \end{enumerate}
      \item \hyperref[sec:streams]{Streams}
      \item \hyperref[sec:procedure]{Procedure calls}
    \end{enumerate}
    \item \hyperref[sec:expressions]{Expressions}
    \begin{enumerate}[label*=.\arabic*]
      \item unary+, unary-, not
      \item \^{}
      \item *,/,\%
      \item +,-
      \item <,>,<=,>=
      \item and
      \item or, xor
      \item Variable refferences
      \item Literal Values
      \item Tuple reference
      \item Function calls
    \end{enumerate}
  \end{enumerate}
\section{Compiler Implementation --- Part 2}

  This section list the elements of the \textit{Gazprea} specification that must be completed for the Part 2 of the compiler implementation. All the elements of Part 1 must have been completed because Part 2 builds on Part 1.

  \begin{enumerate}[label*=\arabic*]
    \item \hyperref[sec:part1]{All Previous Features}
    \item \hyperref[sec:types]{Data Types}
    \begin{enumerate}[label*=.\arabic*]
      \item \hyperref[sec:interval]{Intervals}
      \item \hyperref[sec:vector]{Vectors}
      \item \hyperref[sec:matrix]{Matrices}
      \item \hyperref[sec:string]{Strings}
    \end{enumerate}
    \item \hyperref[sec:statements]{Statements}
    \begin{enumerate}[label*=.\arabic*]
      \item \hyperref[sec:iteratorLoop]{Iterator Loops}
    \end{enumerate}
    \item \hyperref[sec:expressions]{Expression}
    \begin{enumerate}[label*=.\arabic*]
      \item \hyperref[sec:operators]{Operators}
      \item \hyperref[sec:generators]{Generator}
      \item \hyperref[sec:filters]{Filters}
    \end{enumerate}
    \item \hyperref[sec:builtIn]{Built in Functions}
    \begin{enumerate}[label*=.\arabic*]
      \item \hyperref[sec:reverse]{reverse}
      \item \hyperref[sec:rowsColumns]{rows}
      \item \hyperref[sec:rowsColumns]{columns}
      \item \hyperref[sec:length]{length}
      \item \hyperref[sec:streamState]{Stream State}
    \end{enumerate}
    \item \hyperref[sec:memory]{Memory Management}
  \end{enumerate}

\section{Assertions}
\label{sec:assertions}
\subfile{sections/assertions.tex}

\section{Change log}
\label{sec:changelog}
\subfile{sections/changelog.tex}

\end{document}
