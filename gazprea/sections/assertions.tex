\documentclass[../gazprea.tex]{subfiles}

\begin{document}
\textbf{ALL} input test cases will be valid. It can be a good idea to do error checking for your
own testing and debugging, but it is \textit{not necessary}. If you encounter what you think is
undefined behaviour or think something is ambiguous then \textit{do} make a forum post about it to
clarify.

What does it mean to be valid input? The input must adhere to the specification. The rules below
give more in-depth explanation of specification particulars.
\begin{enumerate}
  \item
    \assertiondest{opened-comments}
    Block comments will be opened. Not opening a block comment will cause ANTLR to encounter a lost
    \code{*/} as well as causing a stream of word tokens starting from where the \code{/*} token
    should be. For example, the following tests would be considered invalid:
    \begin{lstlisting}
      int i = 0;
      I'm a closed comment, but I'm never opened... */
      int j = i + 1;
    \end{lstlisting}
  \item
    \assertiondest{closed-comments}
    Block comments will be closed. Not closing a block comment will have ANTLR consume every token
    in the file, rendering the rest of the file useless. For example, the following tests would be
    considered invalid:
    \begin{lstlisting}
      int i = 0;
      /* I'm an opened comment, but I'm never closed...
      int j = i + 1;
    \end{lstlisting}
    % \assertiondest{vector-length}
    % All vectors will have length $l$ such that $0 \leq l \leq 2^{32}$. Trying to create an index
    % greater than $2^{32} - 1$ will cause overflow and result in a negative number. Indexing with a
    % negative number returns $0$. Therefore, vector locations greater than $2^{32} - 1$ would be
    % inaccessible. For example, the following tests would be considered invalid:
    % \begin{lstlisting}
    %   print((0-1)..2147483647);
    % \end{lstlisting}
    %
    % But the following test is valid because the vector length is still within range:
    % \begin{lstlisting}
    %   print((0-2)..2147483645);
    % \end{lstlisting}
\end{enumerate}

\end{document}
