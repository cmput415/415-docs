\documentclass[streams.tex]{subfiles}

\begin{document}
Output streams use the following syntax:
\begin{lstlisting}
  <exp> -> out;
\end{lstlisting}

Values of the following base types are treated as follows when sent to an output stream:
\begin{itemize}
  \item
    \code{character}: The character is printed.
  \item
    \code{integer}: Converted to a string representation, and then printed.
  \item
    \code{real}: Converted to a string representation, and then printed. This is the same behaviour
    as the \href{http://www.cplusplus.com/reference/cstdio/printf/	}{\%g specifier in printf}.
  \item
    \code{boolean}: Print T for true, and F for false.
\end{itemize}

Vectors print their contents according to the rules above, with spaces only \textit{between} values.
For example:
\begin{lstlisting}
  integer vector v = 1..3;
  v -> out;
\end{lstlisting}

prints the following:
\begin{lstlisting}
  1 2 3
\end{lstlisting}

Note that there is \textbf{no automatic new line printed.} To print a new line, a user must
explicitly print the new line character. For example:
\begin{lstlisting}
  '\n' -> out;
\end{lstlisting}

If \texttt{null} or \texttt{identity} is sent to a stream then the result is a null or identity
character being printed.

No other type may be sent to a stream. For instance functions, procedures, and tuples cannot be sent
to streams.
\end{document}
