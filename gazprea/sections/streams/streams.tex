\documentclass[../../gazprea.tex]{subfiles}

\begin{document}
\textit{Gazprea} has two streams: stdout and stdin. They are not automatically available and must be
assigned to a \code{var} as so:
\begin{lstlisting}
  var out = std_output();
  var inp = std_input();
\end{lstlisting}

\subsection{Output Stream}
\label{ssec:output}
\subfile{sections/streams/output.tex}

\subsection{Input Stream}
An input stream is indicated by the following syntax:

\begin{lstlisting}
	<L-value> <- inp;
\end{lstlisting}

The L-value may be anything that can appear on the left hand side of an assignment statement.

Input streams may only work on the following base types:
\begin{itemize}
	\item \texttt{character}: Reads a single character from stdin.
	\item \texttt{integer}: Reads from stdin and tries to parse an integer, if it is not an integer then see the
	error handling section.
	\item \texttt{real}: Reads from stdin and tries to parse a real value, if it is not a real value then see
	the error handling section.
	\item \texttt{boolean}: Reads from stdin and tries to parse a boolean, if it is
	not a boolean value then see the error handling section.
\end{itemize}

\texttt{integer} and \texttt{real} inputs expect the same format used by \textit{Gazprea} literals, and they may
be either negative or positive. These numbers may begin with a single negative sign, or a single positive sign.
\texttt{integer, real,} and \texttt{boolean} expect whitespace between different values in the input.

\texttt{boolean} inputs expect T, or F. Whitespace is expected to separate These values.


\subsubsection{Error Handling}
When reading in values from stdin of certain types it is possible that an error is encountered, or that the end
of the stream has been encountered. In order to handle these situations the language provides a built in
function:

\begin{lstlisting}
	stream_state(inp)
\end{lstlisting}

This function can only be called with the input stream as a parameter, but it's general enough that it could be
used if the language were expanded to include multiple input streams.
When called \texttt{stream\_state} will return an integer value. The return value is an error code defined as
follows:
\begin{itemize}
	\item 0: Last read from the stream was successful
	\item 1: Last read from the stream encountered an error.
	\item 2: Last read from the stream encountered the end of the stream.
\end{itemize}

When an error is encountered the value assigned to the variable used in the stream read operation is
\texttt{null}. When an end of stream is encountered \texttt{null} is assigned to variables of all types except
for character. In this case character is assigned the ASCII EOF value.

When a read from the stream is not successful, and \texttt{stream\_state} returns non-0, then the stream must be
rewound to where it was before the read started.

Reading a character never causes an error. The character will either be successfully read, or the end of the
stream has been reached and character is assigned EOF for each subsequent read.

When reading anything else if something is read which does not match the data type (for instance if the letter
'a' is read while trying to parse an integer), then this is an error. In this case the value returned by the
stream is \texttt{null}, and \texttt{stream\_state} should return 1. Or, if only whitespace and the end of the
stream is encountered the stream is rewound to where it was before the read, \texttt{null} is returned, and
\texttt{stream\_state} will return 2.


\end{document}
