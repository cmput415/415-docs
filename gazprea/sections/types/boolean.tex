\documentclass[types.tex]{subfiles}

\begin{document}
A \code{boolean} is either \code{true} or \code{false}. A \code{boolean} can be represented by an
\code{i1} in \textit{LLVM IR}.

\subsubsection{Declaration}
\label{sssec:boolean_decl}
A \code{boolean} value is declared with the keyword \code{boolean}.

\subsubsection{Null}
\label{sssec:boolean_null}
\code{null} is \code{false} for \code{boolean}.

\subsubsection{Identity}
\label{sssec:boolean_ident}
\code{identity} is \code{true} for \code{boolean}.

\subsubsection{Literals}
\label{sssec:boolean_lit}
The following are the only two valid \code{boolean} literals:
\begin{itemize}
  \item \code{true}
  \item \code{false}
\end{itemize}

\subsubsection{Operations}
\label{sssec:boolean_ops}
The following operations are defined between \code{boolean} values. In all of the usage examples
\code{bool-expr} means some \code{boolean} yielding expression.

\begin{center}
\begin{tabular}{| l | c | l | c |}
  \hline
  \multicolumn{1}{|c|}{\textbf{Operation}} & \textbf{Symbol} & \multicolumn{1}{|c|}{\textbf{Usage}}
    & \textbf{Associativity} \\
  \hline
  negation    & \code{not} & \code{not bool-expr}           & right \\ \hline
  logical or  & \code{or}  & \code{bool-expr or bool-expr}  & left  \\ \hline
  logical xor & \code{xor} & \code{bool-expr xor bool-expr} & left  \\ \hline
  logical and & \code{and} & \code{bool-expr and bool-expr} & left  \\ \hline
  equals      & \code{==}  & \code{bool-expr == bool-expr}  & left  \\ \hline
  not equals  & \code{!=}  & \code{bool-expr != bool-expr}  & left  \\
  \hline
\end{tabular}
\end{center}

Unlike many languages the \code{and} and \code{or} operators do not
\href{https://en.wikipedia.org/wiki/Short-circuit_evaluation}{short circuit evaluation}. Therefore,
both the left hand side and right hand side of an expression must always be evaluated.

This table specifies \code{boolean} operator precedence. Operators without lines between them have
the same level of precedence.
\begin{center}
\begin{tabular}{| c | c |}
  \hline
  \textbf{Precedence} & \textbf{Operation} \\
  \hline
  HIGHER & \code{not} \\ \cline{2-2}
         & ==  \\
         & !=  \\ \cline{2-2}
         & and \\ \cline{2-2}
         & or  \\
  LOWER  & xor \\
  \hline
\end{tabular}
\end{center}

\end{document}
