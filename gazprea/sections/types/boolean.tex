\documentclass[../gazprea.tex]{subfiles}

\begin{document}
Boolean values can either be \texttt{true} or \texttt{false}. In the \textit{LLVM IR} boolean values
should be represented with the type \texttt{i1}.

\subsubsection{Declaration}
\label{sssec:boolean_decl}
Boolean values are declared with the keyword \texttt{boolean}.

\subsubsection{null}
\label{sssec:boolean_null}
\texttt{null} is \texttt{false} for booleans.

\subsubsection{identity}
\label{sssec:boolean_ident}
\texttt{identity} is \texttt{true} for booleans.

\subsubsection{Literals}
\label{sssec:boolean_lit}
Booleans have the following literals:
\begin{itemize}
	\item \texttt{true}
	\item \texttt{false}
\end{itemize}

\subsubsection{Operations}
The following operations are defined between boolean values. In all of the usage examples
\code{bool-expr} means some boolean yielding expression.

\begin{center}
\begin{tabular}{|l|l|l|}
	\hline
	\textbf{Operation} & \textbf{Symbol} & \textbf{Usage} \\
	\hline
	negation    & \texttt{not} & \texttt{not bool-expr}           \\
	logical or  & \texttt{or}  & \texttt{bool-expr or bool-expr}  \\
	logical xor & \texttt{xor} & \texttt{bool-expr xor bool-expr} \\
	logical and & \texttt{and} & \texttt{bool-expr and bool-expr} \\
	equals      & \texttt{==}	 & \texttt{bool-expr == bool-expr}  \\
	not equals  & \texttt{!=}	 & \texttt{bool-expr != bool-expr}  \\
	\hline
\end{tabular}
\end{center}

	Unlike many languages the \texttt{and} and \texttt{or} operators do not short circuit evaluation. You must
	evaluate both the left hand side, and right hand side of these expressions. The following table specifies the
	precedence of these operations:

	\begin{center}
		\begin{tabular}{|l|l|}
			\hline
			\textbf{Precedence} & \textbf{Operations} \\
			\hline
			HIGHER & \texttt{not} \\
			       & ==, !=       \\
			       & or, xor      \\
			LOWER  & and          \\
			\hline
		\end{tabular}
	\end{center}

	All of these operations are left associative.
\end{document}
