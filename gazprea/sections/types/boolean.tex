\documentclass[types.tex]{subfiles}

\begin{document}
Boolean values can either be \code{true} or \code{false}. A boolean can be represented by an
\code{i1} in \textit{LLVM} IR.

\subsubsection{Declaration}
\label{sssec:boolean_decl}
Boolean values are declared with the keyword \code{boolean}.

\subsubsection{Null}
\label{sssec:boolean_null}
\code{null} is \code{false} for booleans.

\subsubsection{Identity}
\label{sssec:boolean_ident}
\code{identity} is \code{true} for booleans.

\subsubsection{Literals}
\label{sssec:boolean_lit}
Booleans have the following literals:
\begin{itemize}
  \item \code{true}
  \item \code{false}
\end{itemize}

\subsubsection{Operations}
\label{sssec:boolean_ops}
The following operations are defined between boolean values. In all of the usage examples
\code{bool-expr} means some boolean yielding expression.

\begin{center}
\begin{tabular}{| l | c | l | c |}
  \hline
  \multicolumn{1}{|c|}{\textbf{Operation}} & \textbf{Symbol} & \multicolumn{1}{|c|}{\textbf{Usage}}
    & \textbf{Associativity} \\
  \hline
  negation    & \code{not} & \code{not bool-expr}           & right \\ \hline
  logical or  & \code{or}  & \code{bool-expr or bool-expr}  & left  \\ \hline
  logical xor & \code{xor} & \code{bool-expr xor bool-expr} & left  \\ \hline
  logical and & \code{and} & \code{bool-expr and bool-expr} & left  \\ \hline
  equals      & \code{==}  & \code{bool-expr == bool-expr}  & left  \\ \hline
  not equals  & \code{!=}  & \code{bool-expr != bool-expr}  & left  \\
  \hline
\end{tabular}
\end{center}

Unlike many languages the \code{and} and \code{or} operators do not
\href{https://en.wikipedia.org/wiki/Short-circuit_evaluation}{short circuit evaluation}. Therefore,
both the left hand side and right hand side of an expression must always be evaluated.

This table specifies boolean operator precedence. Operators without lines between them have the same
level of precedence.
\begin{center}
\begin{tabular}{|c|c|}
  \hline
  \textbf{Precedence} & \textbf{Operation} \\
  \hline
  HIGHER & \code{not} \\ \cline{2-2}
         & == \\
         & != \\ \cline{2-2}
         & or \\
         & xor \\ \cline{2-2}
  LOWER  & and \\
  \hline
\end{tabular}
\end{center}

\end{document}
