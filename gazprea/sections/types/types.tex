\documentclass[../../gazprea.tex]{subfiles}

\begin{document}

\subsection{Boolean}
\label{ssec:boolean}
\subfile{sections/types/boolean.tex}

\subsection{Character}
\label{ssec:character}
\subfile{sections/types/character.tex}

\subsection{Integer}
\label{ssec:integer}
\subfile{sections/types/integer.tex}

\subsection{Real}
\label{ssec:real}
\subfile{sections/types/real.tex}

\subsection{Tuple}
\label{ssec:tuple}
\subfile{sections/types/tuple.tex}

\subsection{Interval}
\label{ssec:interval}
\subfile{sections/types/interval.tex}

	\subsection{Vectors}\label{sec:vector}

		Vectors are arrays that can contain any of the following base types:
		\begin{itemize}
			\item \texttt{boolean}
			\item \texttt{integer}
			\item \texttt{real}
			\item \texttt{character}
		\end{itemize}

		In \textit{Gazprea} the number of elements in the vector also determine its type. A 3 element vector of any base
		type is always considered a diffent type form a 2 element vector.

		\subsubsection{Declaration}

			Vectors may be declared multiple ways. Aside from any type specifiers, the base type of the vector is the
			first portion of the declaration. A vector may be declared with the \texttt{vector} keyword, using square
			brackets much like arrays in \textit{C}, or a combination of both.

			If possible, initialization expressions may go through an implicit type conversion. For instance, when
			declaring a real vector if it is initialized with an integer value the integer will be promoted to a real
			value, and then used as a scalar initialization of the vector.

			\begin{enumerate}
				\item Explicit Size Declarations

					When a vector is declared it may be explicitly given a size. This size can be given as any integer
					expression, thus the size of the vector may not be known until runtime.

					\begin{lstlisting}
						<type> <identifier>[<int-expr>];
						<type> <identifier>[<int-expr>] = <type-expr>;
						<type> <identifier>[<int-expr>] = <type-vector>;
					\end{lstlisting}

					The size of the vector is given by the integer expression between the square brackets.

					In any of these cases the vector keyword may also be used, but its use does not change anything.

					\begin{lstlisting}
						<type> vector <identifier>[<int-expr>];
						<type> vector <identifier>[<int-expr>] = <type-expr>;
						<type> vector <identifier>[<int-expr>] = <type-vector>;
					\end{lstlisting}

					If the vector is given a scalar value of the same base type then the scalar value is duplicated for
					every single element of the vector.

					A vector may also be initialized with another vector. If the vector is initialized using a vector
					that is too small then the vector will be null padded. However, if the vector is initialized with a
					vector that is too large then a type error will occur.

				\item Inferred Size Declarations

					If a vector is assigned an initial value when it is declared, then its size may be inferred. There
					is no need to repeat the size in the declaration because the size of the vector on the right-hand
					side is known. Again, there are several different syntaxes for these declarations:

					\begin{lstlisting}
						<type> <identifier>[*] = <type-vector>;
						<type> vector <identifier> = <type-vector>;
						<type> vector <identifier>[*] = <type-vector>;
					\end{lstlisting}

				\item Inferred Type and Size

					It is also possible to declare a vector with an implied type and length using the var keyword. This
					type of declaration can only be used when the variable is initialized in the declaration, otherwise
					the compiler will not be able to infer the type or the size of the vector.

					\begin{lstlisting}
						integer vector v = [ 1, 2, 3];
						var w = v + 1;
					\end{lstlisting}
					In this example the compiler can infer both the size and the type of w from v. The size may not
					always be known at compile time, so this may need to be handled during runtime.

			\end{enumerate}

		\subsubsection{null}

			Vector of \texttt{null} elements.

			When initializing a vector to a value of \texttt{null} an explicit size must be given. Such initialization
			is equivalent to promoting a \texttt{null} value of the base type to the vector.

		\subsubsection{identity}

			Vector of \texttt{identity} elements.

			When initializing a vector to a value of \texttt{identity} an explicit size must be given. Such
			initialization is equivalent to promoting a \texttt{identity} value of the base type to the vector.

		\subsubsection{Construction}

			A vector value in \textit{Gazprea} may be constructed using the following notation:

			\begin{lstlisting}
				[ expr1, expr2, ..., exprN]
			\end{lstlisting}

			Each \texttt{expK} is an expression with a compatible type. In the simplest cases each expression is of the
			same type, but it is possible to mix the types as long as all of the types can be promoted to a common type.
			For instance it is possible to mix integers and real numbers.

			\begin{lstlisting}
				real vector v = [1, 3.3, 5 * 3.4];
			\end{lstlisting}

			It is also possible to construct a single-element vector using this method of construction.

			\begin{lstlisting}
				real vector v = [7];
			\end{lstlisting}

			\textit{Gazprea} \textbf{DOES} support empty vectors.

			\begin{lstlisting}
				real vector v = []; /* Should create an empty vector */
			\end{lstlisting}

		\subsubsection{Operations}

			\begin{enumerate}
				\item Vector Operations and functions
					\begin{enumerate}
						\item length
							The number of elements in a vector is given by the built-in functions \texttt{length}. For
							instance:

							\begin{lstlisting}
								integer vector v = [8, 9, 6];
								integer numElements = length(v);
							\end{lstlisting}

							In this case \texttt{numElements} would be 3, since the vector v contains 3 elements.

						\item Concatenation

							Two vectors with the same base type may be concatenated into a single vector using the
							concatenation operator, \texttt{||}. For instance:

							\begin{lstlisting}
								[1, 2, 3] || [4, 5] // produces [1, 2, 3, 4, 5]
								[1, 2] || [] || [3, 4] // produces [1, 2, 3, 4]
							\end{lstlisting}

							Concatenation is also allowed between vectors of different base types, as long as one base
							type is coerced automatically to the other. For instance:

							\begin{lstlisting}
								integer v[3] = [1, 2, 3];
								real u[3] = [4.0, 5.0, 6.0];
								real j[6] = v || u;
							\end{lstlisting}

							would be permitted, and the integer vector v would be promoted to a real vector before the
							concatenation.

							Concatenation may also be used with scalar values. In this case the scalar values are
							treated as though they were single element vectors.

							\begin{lstlisting}
								[1, 2, 3] || 4 // produces [1, 2, 3, 4]
								1 || [2, 3, 4] // produces [1, 2, 3, 4]
							\end{lstlisting}

						\item Dot Product

							Two vectors with the same size and a numeric base type(types with the +, and * operator)
							may be used in a dot product operation. For instance:

							\begin{lstlisting}
								integer v[3] = [1, 2, 3];
								integer u[3] = [4, 5, 6];

								/* v[1] * u[1] + v[2] * u[2] + v[3] * u[3] */
								/* 1 * 4 + 2 * 5 + 3 * 6 &=&  32 */
								integer dot = v ** u;  /* Perform a dot product */
							\end{lstlisting}

						\item Indexing

							A vector may be indexed in order to retrieve the values stored in the vector. A vector may
							be indexed using integers, integer vectors, and integer intervals. \textit{Gazprea} is
							1-indexed, so the first element of a vector is at index 1 (as opposed to index 0 in
							languages like \textit{C}).  For instance:

							\begin{lstlisting}
								intger vector v[3] = [4, 5, 6];

								integer x = v[2]; /* x ==  5 */
								integer vector y = v[2..3]; /* y ==  [5, 6] */
								integer vector z = v[[ 3, 1, 2]]; /* z ==  [6, 4, 5] */
							\end{lstlisting}

							When indexed with a scalar integer the result is a scalar value, but when indexed with an
							interval or a vector the result is another vector.

							Out of bounds indexing should cause an error.

						\item by

							The by operator is also defined for vectors of any base type. It produces a vector with
							every value with the given offset. For instance:

							\begin{lstlisting}
								integer vector v = 1..5 by 1; /* [1, 2, 3, 4, 5] */
								integer vector u = v by 1; /* [1, 2, 3, 4, 5] */
								integer vector w = v by 2; /* [1, 3, 5] */
								integer vector l = v by 3; /* [1, 4] */
							\end{lstlisting}
					\end{enumerate}

				\item Operations of the Base Type

					Unary operations that are valid for the base type of a vector may be applied to the vector in order
					to produce a vector whose result is the equivalent to applying that unary operation to each element
					of the vector. For instance:

					\begin{lstlisting}
						boolean vector v = [true, false, true, true];
						boolean vector nv = not v;
					\end{lstlisting}

					\texttt{nv} would have a value of \texttt{[not true, not false, not true, not true] = [false, true,
					false, false]}.

					Similarly most binary operations that are valid to the base type of a vector may be also applied to
					two vectors. When applied to two vectors of the same size, the result of the binary operation is a
					vector formed by the element-wise application of the binary operation to the vector operands.

					\begin{lstlisting}
						[1, 2, 3, 4] + [2, 2, 2, 2] // results in [3, 4, 5, 6]
					\end{lstlisting}

					Attempting to perform a binary operation between two vectors of different sizes should result in a
					type error.

					When one of the operands of a binary operation is a vector and the other operand this a scalar
					value, then the scalar value must first be promoted with a vector of the same size as the vector
					operand and with the value of each element equal the scalar value. For example:

					\begin{lstlisting}
						[1, 2, 3, 4] + 2 // results in [3, 4, 5, 6]
					\end{lstlisting}

					Additionally the base types of vectors may be promoted, for instance in this case the integer vector
					must be promoted to a real vector in order to perform the operation:

					\begin{lstlisting}
						[1, 2, 3, 4] + 2.3 // results in [3.3, 4.3, 5.3, 6.3]
					\end{lstlisting}

					The equality operation is the exception to the behavior of the binary operations. Instead of
					producing a boolean vector, an equality operation checks whether or not all of the elements of two
					vectors are equal, and return a single boolean value reflecting the result of this comparison.

					\begin{lstlisting}
						[1, 2, 3] == [1, 2, 3]
					\end{lstlisting}

					yields \texttt{true}

					\begin{lstlisting}
						[1, 1, 3] == [1, 2, 3]
					\end{lstlisting}

					yields \texttt{false}

					The != operation also produces a boolean instead of a boolean vector. The result is the logical
					negation of the result of the == operator.
			\end{enumerate}


	\subsection{String}\label{sec:string}

		A string is just a type synonym for a vector of characters.

		\subsubsection{Declaration}

			A string may be declared with the type \texttt{string. string} is a built-n typedef for character vector.

		\subsubsection{null}

			Same behaviour as \texttt{null} for vectors. Vector filled with \texttt{null} characters.

		\subsubsection{identity}

			Same behaviour as \texttt{identity} for vectors. Vector filled with \texttt{identity} characters.

		\subsubsection{Literals}

			Strings can be constructed in the same way as vectors using character literals.  \textit{Gazprea} also
			provides a special syntax for string literals. A string literal is any sequence of character literals
			(including escape sequences) in between double quotes. For instance:

			\begin{lstlisting}
				string cats_meow = "The cat said \"Meow!\"\nThat was a good day.\n"
			\end{lstlisting}

		\subsubsection{Operations}

			Strings have all of the same operations defined on them as the other vector data types.


	\subsection{Matrix}\label{sec:matrix}

		\textit{Gazprea} supports two dimensional matrices. A matrix can have all of the same base types a vector can:

		\begin{itemize}
			\item \texttt{boolean}
			\item \texttt{integer}
			\item \texttt{real}
			\item \texttt{character}
		\end{itemize}

		\subsubsection{Declaration}

			A matrix is declared using the \texttt{matrix} keyword. Matrix declarations are similar to vector
			declarations, the difference being that matrices have two dimensions instead of one. The following are
			valid matrix declarations:

			\begin{lstlisting}
				integer matrix A = [[1, 2, 3], [4, 5, 6], [7, 8, 9]];
				integer matrix A[3, 2] = [[1, 2], [4, 5], [7, 8]];
				integer matrix A[3, *] = [[1, 2], [4, 5], [7, 8]];
				integer matrix A[*, 2] = [[1, 2], [4, 5], [7, 8]];
				integer matrix A[*, *] = [[1, 2], [4, 5], [7, 8]];
			\end{lstlisting}

		\subsubsection{null}

			Matrix of \texttt{null} elements.

		\subsubsection{identity}

			Matrix of \texttt{identity} elements.

		\subsubsection{Construction}

			To construct a matrix the programmer may use nested vectors. Each vector element represents a single row of
			the matrix. All rows with fewer elements than the row of maximum row length are padded with \texttt{null}
			values on the right. Similarly, if the matrix is declared with a column length larger than the number of
			rows provided, the bottom rows of the matrix are \texttt{null}.  If the number of rows or columns exceeds
			the amounts given in a declaration an error is to be produced.

			\begin{lstlisting}
				integer vector v = [1, 2, 3];
				integer matrix A = [v, [1, 2]];
				/* A == [[1, 2, 3], [1, 2, 0]] */
			\end{lstlisting}

			Similarly, we can have:

			\begin{lstlisting}
				integer vector v = [1, 2, 3];
				integer matrix A[3, 3] = [v, [1, 2]];
				/* A == [[1, 2, 3], [1, 2, 0], [0, 0, 0]] */
			\end{lstlisting}

			Also matrices can be initialized with a scalar value, \texttt{null}, or \texttt{identity}. \texttt{null} and
			\texttt{identity} behave as previously described. Initializing with a scalar value makes every element of
			the matrix equal to the scalar.

		\subsubsection{Operations}

			Matrices have binary and unary operations of the base type defined in the same manner as vectors. Unary
			operations are applied to every element of the matrix, and binary operations are applied between elements
			with the same position in two matrices.

			The operators ==, and != also have the same behaviors that vectors do. These operations compare whether or
			not \textbf{all} elements of two matrices are equal.

			In addition to this matrices have several special operations defined on them.  If the base type is numeric
			(supports addition, and multiplication), then matrix multiplication is supported using the operator **.
			Matrix multiplication is only defined between matrices with compatible base types, and the dimensions of the
			matrices must be valid for performing a matrix multiplication. If this is not the case then an error should
			be raised.

			All matrices support the built in functions \texttt{rows} and \texttt{columns}, which when passed a matrix
			yields the number of rows and columns in the matrix respectively. For instance:

			\begin{lstlisting}
				integer matrix M = [[1, 1, 1], [1, 1, 1]];

				integer r = rows(M);  /* This has a value of 2 */
				integer c = columns(M);  /* This has a value of 3 */
			\end{lstlisting}

			Matrix indexing is done similarly to vector indexing, however, two indices must be used. These indices are
			separated using a comma.

			\begin{lstlisting}
				M[i, j] -> out;
			\end{lstlisting}

			The first index specifies the row of the matrix, and the second index specifies the column of the matrix.
			The result is retrieved from the row and column.  Both the row and column indices can be either integers,
			integer intervals, or integer vectors.  When both indices are scalar integers the result is the scalar value
			in the row and column specified.

			\begin{lstlisting}
				integer matrix M = [[11, 12, 13], [21, 22, 23]];

				/* M[1, 2] == 12 */
			\end{lstlisting}

			If one of the indices is an interval or a vector, and the other index is a scalar, then the result is a
			vector. For example:

			\begin{lstlisting}
				integer matrix M = [[11, 12, 13], [21, 22, 23]];

				/* Select from row 2 */
				/* M[2, 2..3] == [22, 23] */
				/* M[2, [3, 2]] == [23, 22] */

				/* Select from column 1 */
				/* M[1..2, 1] == [11, 21] */
				/* M[[2, 1], 1] == [21, 11] */
			\end{lstlisting}

			Finally, both of the indices may be intervals or vectors, in which case the result is another matrix.

			\begin{lstlisting}
				integer matrix M = [[11, 12, 13], [21, 22, 23]];

				/* Makes a matrix consisting of [[M[2, 1], M[2, 3]], [M[1, 1], M[1, 3]]] */
				integer matrix K = M[[2, 1], [1, 3]];
			\end{lstlisting}

			As with vectors, out of bounds indexing is an error on Matrices.


\end{document}
