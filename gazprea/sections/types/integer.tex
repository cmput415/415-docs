\documentclass[types.tex]{subfiles}

\begin{document}
An \code{integer} is a signed 32-bit value. An \code{integer} can be represented by an \code{i32} in
\textit{LLVM IR}.

\subsubsection{Declaration}
\label{sssec:integer_decl}
A \code{integer} value is declared with the keyword \code{integer}.

\subsubsection{Null}
\label{sssec:integer_null}
\code{null} is \code{0} for \code{integer}.

\subsubsection{Identity}
\label{sssec:integer_ident}
\code{identity} is \code{1} for \code{integer}.

\subsubsection{Literals}
\label{sssec:int_lit}
An \code{integer} literal is specified in base 10. For example:
\begin{lstlisting}
  1234
  2
  0
\end{lstlisting}

To aid in the readability of large numbers, underscores may be inserted anywhere within or at the
end of an \code{integer} literal as a separator. For example, the following literals would produce
the same \code{integer} value:
\begin{lstlisting}
  1__23_4
  1234
  1_2_3_4___
\end{lstlisting}

The underscore may \textbf{NOT} appear at the beginning of the \code{integer} literal because then
it would be recognised as an identifier. For example, the following would be identifiers and
\textit{not} \code{integer} literals:
\begin{lstlisting}
  _2
  ____2_3
  _2__3_
\end{lstlisting}

\subsubsection{Operations}
The following operations are defined between \code{integer} values. In all of the usage examples
\code{int-expr} means some \code{integer} yielding expression.

\begin{center}
  \begin{tabular}{| c | l | c | l | c |}
    \hline
    \textbf{Class} & \multicolumn{1}{|c|}{\textbf{Operation}} & \textbf{Symbol} &
    \multicolumn{1}{|c|}{\textbf{Usage}} & \textbf{Associativity} \\
    \hline
    Arithmetic & addition           & \code{+}  & \code{int-expr + int-expr}  & left  \\
               & subtraction        & \code{-}  & \code{int-expr - int-expr}  & left  \\
               & multiplication     & \code{*}  & \code{int-expr * int-expr}  & left  \\
               & division           & \code{/}  & \code{int-expr / int-expr}  & left  \\
               & remainder          & \code{\%} & \code{int-expr \% int-expr} & left  \\
               & exponentiation     & \code{^}  & \code{int-expr ^ int-expr}  & right \\
               & unary negation     & \code{-}  & \code{- int-expr} & right \\
               & unary plus (no-op) & \code{+}  & \code{+ int-expr} & right \\
    \hline
    Comparison & less than                & \code{<}  & \code{int-expr < int-expr}  & left \\
               & greater than             & \code{>}  & \code{int-expr > int-expr}  & left \\
               & less than or equal to    & \code{<=} & \code{int-expr <= int-expr} & left \\
               & greater than or equal to & \code{>=} & \code{int-expr >= int-expr} & left \\
               & equals                   & \code{==} & \code{int-expr == int-expr} & left \\
               & not equals               & \code{!=} & \code{int-expr != int-expr} & left \\
    \hline
  \end{tabular}
\end{center}

Unary negation produces the additive inverse of the \code{integer} expression. Unary plus always
produces the same result as the \code{integer} expression it is applied to. Remainder mirrors the
behaviour of remainder in \textit{C99}.

This table specifies \code{integer} operator precedence. Operators without lines between them have
the same level of precedence.
\begin{center}
\begin{tabular}{| c | c |}
  \hline
  \textbf{Precedence} & \textbf{Operations} \\
  \hline
  HIGHER & \code{unary +} \\
         & \code{unary -} \\ \cline{2-2}
         & \code{^} \\ \cline{2-2}
         & \code{*}  \\
         & \code{/}  \\
         & \code{\%} \\ \cline{2-2}
         & \code{+} \\
         & \code{-} \\ \cline{2-2}
         & \code{<}  \\
         & \code{>}  \\
         & \code{<=} \\
         & \code{>=} \\ \cline{2-2}
         & \code{==} \\
  LOWER  & \code{!=} \\
  \hline
\end{tabular}
\end{center}

\end{document}
