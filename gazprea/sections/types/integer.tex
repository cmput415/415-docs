\documentclass[types.tex]{subfiles}

\begin{document}
Integers are signed 32-bit values. An integer can be represented by an \code{i32} in \textit{LLVM}
IR.

\subsubsection{Declaration}
\label{sssec:integer_decl}
Integer values are declared with the keyword \code{integer}.

\subsubsection{Null}
\label{sssec:integer_null}
\code{null} is \code{0} for integers.

\subsubsection{Identity}
\label{sssec:integer_ident}
\code{identity} is \code{1} for integers.

\subsubsection{Literals}
\label{sssec:int_lit}
Integer literals are specified in base 10. For example:
\begin{lstlisting}
  1234
  2
  0
\end{lstlisting}

To aid in the readability of large numbers, underscores may be inserted anywhere within or at the
end of an integer literal as a separator. For example, the following literals would produce the same
integer value:
\begin{lstlisting}
  1__23_4
  1234
  1_2_3_4___
\end{lstlisting}

The underscore may \textbf{NOT} appear at the beginning of the integer literal because then it would
be recognised as an identifier. For example, the following would be identifiers and \textit{not}
integer literals:
\begin{lstlisting}
  _2
  ____2_3
  _2__3_
\end{lstlisting}

\subsubsection{Operations}
The following operations are defined on integer values. In all of the usage examples \code{int-expr}
means some integer yielding expression.

\begin{center}
  \begin{tabular}{|c|l|c|l|c|}
    \hline
    \textbf{Class} & \multicolumn{1}{|c|}{\textbf{Operation}} & \textbf{Symbol} &
    \multicolumn{1}{|c|}{\textbf{Usage}} & \textbf{Associativity} \\
    \hline
    Arithmetic & addition           & \code{+}  & \code{int-expr + int-expr}  & left  \\
               & subtraction        & \code{-}  & \code{int-expr - int-expr}  & left  \\
               & multiplication     & \code{*}  & \code{int-expr * int-expr}  & left  \\
               & division           & \code{/}  & \code{int-expr / int-expr}  & left  \\
               & remainder          & \code{\%} & \code{int-expr \% int-expr} & left  \\
               & exponentiation     & \code{^}  & \code{int-expr ^ int-expr}  & right \\
               & unary negation     & \code{-}  & \code{- int-expr} & right \\
               & unary plus (no-op) & \code{+}  & \code{+ int-expr} & right \\
    \hline
    Comparison & less than                & \code{<}  & \code{int-expr < int-expr}  & left \\
               & greater than             & \code{>}  & \code{int-expr > int-expr}  & left \\
               & less than or equal to    & \code{<=} & \code{int-expr <= int-expr} & left \\
               & greater than or equal to & \code{>=} & \code{int-expr >= int-expr} & left \\
               & equals                   & \code{==} & \code{int-expr == int-expr} & left \\
               & not equals               & \code{!=} & \code{int-expr != int-expr} & left \\
    \hline
  \end{tabular}
\end{center}

Unary negation produces the additive inverse of the integer expression. Unary plus always produces
the same result as the integer expression it is applied to. Remainder mirrors the behaviour of
remainder in \textit{C99}.

This table specifies integer operator precedence. Operators without lines between them have the same
level of precedence.
\begin{center}
\begin{tabular}{| c | c |}
  \hline
  \textbf{Precedence} & \textbf{Operations} \\
  \hline
  HIGHER & \code{unary +} \\
         & \code{unary -} \\ \cline{2-2}
         & \code{^} \\ \cline{2-2}
         & \code{*}  \\
         & \code{/}  \\
         & \code{\%} \\ \cline{2-2}
         & \code{+} \\
         & \code{-} \\ \cline{2-2}
         & \code{<}  \\
         & \code{>}  \\
         & \code{<=} \\
         & \code{>=} \\ \cline{2-2}
         & \code{==} \\
  LOWER  & \code{!=} \\
  \hline
\end{tabular}
\end{center}

\end{document}
