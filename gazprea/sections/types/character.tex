\documentclass[types.tex]{subfiles}

\begin{document}
A \code{characters} is a signed 8-bit value. A \code{character} can be represented by an \code{i8}
in \textit{LLVM IR}.

\subsubsection{Declaration}
\label{sssec:character_decl}
A \code{character} value is declared with the keyword \code{character}.

\subsubsection{Null}
\label{sssec:character_null}
\textsf{null} is ASCII \code{NUL} (\code{'\\0'}, \code{0x00}) for \code{character}.

\subsubsection{Identity}
\label{sssec:character_ident}
\textsf{identity} is ASCII \code{SOH} (\code{0x01}) for characters. This choice allows the casting
of a \code{character} to an \code{integer} to yield the \code{integer} \code{identity}.

\subsubsection{Literals and Escape Sequences}
\label{sssec:character_lit}
A \code{character} literal is written in the same manner as \textit{C99}: a single character
enclosed in single quotes. For example:
\begin{lstlisting}
  'a'
  'b'
  'A'
  '1'
  '.'
  '*'
\end{lstlisting}

As in \textit{C99}, \textit{Gazprea} supports character escape sequences for common characters. For
example:
\begin{lstlisting}
  '\0'
  '\n'
\end{lstlisting}

The following escape sequences are supported by \textit{Gazprea}:
\begin{center}
  \begin{tabular}{| l | c | c |}
    \hline
    \textbf{Description} & \textbf{Escape Sequence} & \textbf{Value (Hex)} \\
    \hline
    Null            & \code{\\0}  & \code{0x00} \\
    Bell            & \code{\\a}  & \code{0x07} \\
    Backspace       & \code{\\b}  & \code{0x08} \\
    Tab             & \code{\\t}  & \code{0x09} \\
    Line Feed       & \code{\\n}  & \code{0x0A} \\
    Carriage Return & \code{\\r}  & \code{0x0D} \\
    Quotation Mark  & \code{\\"}  & \code{0x22} \\
    Apostrophe      & \code{\\'}  & \code{0x27} \\
    Backslash       & \code{\\\\} & \code{0x5C} \\
    \hline
  \end{tabular}
\end{center}

\subsubsection{Operations}
\label{sssec:character_ops}
There are no operations defined for a \code{character}. To operate on a \code{character} it must
first be cast to either an \code{boolean}, \code{integer}, or \code{real}.

\end{document}
