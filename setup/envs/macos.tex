\documentclass[../setup.tex]{subfiles}

\begin{document}

\subsection{Installing Oracle Java JRE 8}
\begin{enumerate}
  \item
    Go to
    \href{http://www.oracle.com/technetwork/java/javase/downloads/jre8-downloads-2133155.html} {the
    Oracle download page} and download the latest Mac OS X .dmg.
  \item
    Run the Oracle Java installer follow all the steps.
\end{enumerate}

\subsection{Installing Homebrew}
Homebrew is a package manager for Mac OS X. If you don't have it yet, you can read about it
\href{https://brew.sh/} {here}. Otherwise, install it via the command on their front page:
\begin{lstlisting}
  /usr/bin/ruby -e "$(curl -fsSL https://raw.githubusercontent.com/Homebrew/install/master/install)"
\end{lstlisting}

\subsection{Installing Git}
Install git via homebrew:
\begin{lstlisting}
  brew install git
\end{lstlisting}

	\subsection{Installing Intellij}

		\begin{enumerate}
			\item Go to the \href{https://www.jetbrains.com/idea/download/index.html} {jetbrains} and download
			\textit{Intellij Community} for OSx.
			\item Launch the installer by double clicking on the \textit{.dmg}
			\item Open your \texttt{Applications} and launch Intellij IDEA 14.app to finish the installation
			\begin{enumerate}
				\item Select I don't have any configurations.
				\item Select \texttt{JetBrains account} and use the account information you just created to
				authenticate. \textbf{You must have already authenticated it first otherwise this won't work.}
				\item Pick whichever UI is prettiest to you. Then hit \texttt{Next: Keymaps}
				\item Select which keymap scheme you want. Then hit \texttt{Next: Default Plug-ins}
				\item Leave Java Frameworks and Build tools enabled everything else can be disabled at your discretion.
				Then hit \texttt{Next: Feature Plug-ins}
				\item None of these are necessary so you can hit \texttt{Start using IntellijIDEA}
			\end{enumerate}
		\end{enumerate}

\subsection{Installing the ANTLR Plugin for CLion}
ANTLR has a CLion integration that gives syntax highlighting as well as tool for visualising the
parse tree for a grammar rule and an input.
\begin{enumerate}
	\item
    Launch CLion by going to the application launcher (tap the super/Windows button) and typing
    \lstinline{clion}. This should launch CLion.
	\item
    Open the settings window \texttt{File $\rightarrow$ Settings...}
  \item
    Select \texttt{Plugins} from the menu on the left.
	\item
    Click \texttt{Browse Repositories} below the plugin list.
  \item
    In the new window, type \texttt{antlr} into the search bar at the top.
	\item
    From the list select \lstinline{ANTLR v4 grammar plugin}
  \item
    Click \texttt{Install} in the right pane.
	\item
    After the install bar ends click the \texttt{Restart CLion} button that should have replaced
    the \texttt{Install} button.
\end{enumerate}


	\subsection{Getting Antlr4 libraries}
		\begin{enumerate}
			\item Follow the steps on this
			\href{https://github.com/antlr/antlr4/blob/master/doc/getting-started.md}{tutorial}
			\item Follow the steps on this
			\href
			{https://github.com/antlr/stringtemplate4/blob/master/doc/java.md}
			{tutorial}
		\end{enumerate}


	\subsection{Configuring first project}
		\begin{enumerate}
			\item Click \texttt{Create new project}
			\item Select Java Project and then next to \textit{Project SDK} select \texttt{New} $\rightarrow$ \texttt{JDK}
			\item The installation process will give you an interactive interface (similar to finder) to locate the
			file. At the very top of the window select \texttt{"Computer"}, then follow the directory path Macintosh
			\begin{lstlisting}
				HD/Library/Java/JavaVirtualMachines
			\end{lstlisting}
			and select the folder \texttt{jdk1.8.0\_45.jdk} (or whichever is the current version that you are
			installing).
			\item Click \texttt{Next} then click \texttt{Next} again.
			\item Name your project and click \texttt{Finish}
			\item Open Project manager (Default hotkey CMD-1)
			\item In the project menu that you just opened right-click on the \texttt{src} folder $\rightarrow$ New $\rightarrow$ File
			\item name the file
			\begin{lstlisting}
				hello.g4
			\end{lstlisting}
			\item In the new file put the following content
			\begin{lstlisting}
				grammar hello;

				rule: 'hello' id;
				id  : STRING;

				STRING: CHAR+;
				CHAR: [a-z|A-Z];
				WS: ' '+ -> skip;
			\end{lstlisting}
			\item CLTR-click on \texttt{rule} $\rightarrow$ select \texttt{Test Rule rule}
			\item There are three new boxes that open:
			\begin{enumerate}
				\item a small that appears to be to input a file name
				\item a larger one underneath
				\item To the right of that there is a box on top of which is "Parse tree | Profiler"
			\end{enumerate}
			in the second box type
			\begin{lstlisting}
				hello <your name>
			\end{lstlisting}
			in the third box a tree should appear a parse tree
			\item Congratulations you've made your first Antlr4 parser

		\end{enumerate}

\end{document}
