\documentclass[../setup.tex]{subfiles}

\begin{document}

\subsection{Installing Oracle Java JRE 8}
\begin{enumerate}
  \item
    Go to
    \href{http://www.oracle.com/technetwork/java/javase/downloads/jre8-downloads-2133155.html} {the
    Oracle download page} and download the latest Mac OS X \lstinline{.dmg}.
  \item
    Run the Oracle Java installer follow all the steps.
\end{enumerate}

\subsection{Installing Homebrew}
Homebrew is a package manager for Mac OS X. If you don't have it yet, you can read about it
\href{https://brew.sh/} {here}. Otherwise, install it via the command on their front page:
\begin{lstlisting}
  /usr/bin/ruby -e "$(curl -fsSL https://raw.githubusercontent.com/Homebrew/install/master/install)"
\end{lstlisting}

\subsection{Installing Git}
Install git via homebrew:
\begin{lstlisting}
  brew install git
\end{lstlisting}

\subsection{Installing CLion}
\begin{enumerate}
  \item
    Go to the \href{https://www.jetbrains.com/clion/download/\#section=mac} {download page} and
    download \textit{CLion} for macOs.
  \item
    Execute the installer through the downloaded \lstinline{.dmg} and follow the installation
    wizard.
  \item
    Perform the initial set up of CLion.
    \begin{enumerate}
      \item
        You should be presented with a prompt for your license. Select \texttt{Activate},
        \texttt{JetBrains Account}, enter your UAlberta email address and JetBrains password.
      \item
        Select \texttt{Do not import settings} and click \texttt{OK}.
      \item
        Pick your favorite UI. Then click \texttt{Next: Default Plugins}
      \item
        You might consider disabling all but the git plugin, and even then, using it is up to you.
        It can be useful to see the color coded files for differences at a glance or track changes
        in a file. You should consider disabling all of the web development plugins. Disabling
        other tools is up to you as well. Now select \texttt{Next: Feature Plugins}
      \item
        Again, the choices here are yours if you like vim, then maybe the vim plugin is up your
        alley. The markdown plugin can be useful as well. You do not need the TeamCity Integration.
        Select \texttt{Start using CLion}
    \end{enumerate}
\end{enumerate}

\subsection{Installing the ANTLR Plugin for CLion}
ANTLR has a CLion integration that gives syntax highlighting as well as tool for visualising the
parse tree for a grammar rule and an input.
\begin{enumerate}
  \item
    Launch CLion by going to the application launcher (tap the super/Windows button) and typing
    \lstinline{clion}. This should launch CLion.
  \item
    Open the settings window \texttt{File $\rightarrow$ Settings...}
  \item
    Select \texttt{Plugins} from the menu on the left.
  \item
    Click \texttt{Browse Repositories} below the plugin list.
  \item
    In the new window, type \texttt{antlr} into the search bar at the top.
  \item
    From the list select \lstinline{ANTLR v4 grammar plugin}
  \item
    Click \texttt{Install} in the right pane.
  \item
    After the install bar ends click the \texttt{Restart CLion} button that should have replaced
    the \texttt{Install} button.
\end{enumerate}


  \subsection{Getting Antlr4 libraries}
    \begin{enumerate}
      \item Follow the steps on this
      \href{https://github.com/antlr/antlr4/blob/master/doc/getting-started.md}{tutorial}
      \item Follow the steps on this
      \href
      {https://github.com/antlr/stringtemplate4/blob/master/doc/java.md}
      {tutorial}
    \end{enumerate}


  \subsection{Configuring first project}
    \begin{enumerate}
      \item Click \texttt{Create new project}
      \item Select Java Project and then next to \textit{Project SDK} select \texttt{New} $\rightarrow$ \texttt{JDK}
      \item The installation process will give you an interactive interface (similar to finder) to locate the
      file. At the very top of the window select \texttt{"Computer"}, then follow the directory path Macintosh
      \begin{lstlisting}
        HD/Library/Java/JavaVirtualMachines
      \end{lstlisting}
      and select the folder \texttt{jdk1.8.0\_45.jdk} (or whichever is the current version that you are
      installing).
      \item Click \texttt{Next} then click \texttt{Next} again.
      \item Name your project and click \texttt{Finish}
      \item Open Project manager (Default hotkey CMD-1)
      \item In the project menu that you just opened right-click on the \texttt{src} folder $\rightarrow$ New $\rightarrow$ File
      \item name the file
      \begin{lstlisting}
        hello.g4
      \end{lstlisting}
      \item In the new file put the following content
      \begin{lstlisting}
        grammar hello;

        rule: 'hello' id;
        id  : STRING;

        STRING: CHAR+;
        CHAR: [a-z|A-Z];
        WS: ' '+ -> skip;
      \end{lstlisting}
      \item CLTR-click on \texttt{rule} $\rightarrow$ select \texttt{Test Rule rule}
      \item There are three new boxes that open:
      \begin{enumerate}
        \item a small that appears to be to input a file name
        \item a larger one underneath
        \item To the right of that there is a box on top of which is "Parse tree | Profiler"
      \end{enumerate}
      in the second box type
      \begin{lstlisting}
        hello <your name>
      \end{lstlisting}
      in the third box a tree should appear a parse tree
      \item Congratulations you've made your first Antlr4 parser

    \end{enumerate}

\end{document}
