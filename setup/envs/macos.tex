\documentclass[../setup.tex]{subfiles}

\begin{document}

\subsection{Installing Developer tools}
It's likely that you've already done this since you're in the Computing Science program, but in the
event that you have a fresh install, run the following command to install Mac OS developer tools:
\begin{lstlisting}
  xcode-select --install
\end{lstlisting}

\subsection{Installing Homebrew}
Homebrew is a package manager for Mac OS X. If you don't have it yet, you can read about it
\href{https://brew.sh/} {here}. Otherwise, install it via the command on their front page:
\begin{lstlisting}
  /usr/bin/ruby -e "$(curl -fsSL https://raw.githubusercontent.com/Homebrew/install/master/install)"
\end{lstlisting}
You can check that it succeeded by checking its version:
\begin{lstlisting}
  brew --version
\end{lstlisting}
Homebrew itself is not a requirement, just an easy suggestion, but a package manager is. Using
another package manager (like \href{https://nixos.org/nix/} {Nix}) is fine, as long as you
understand how to install packages.

\subsection{Installing Oracle Java JDK}
Installing from the Oracle download page requires a bunch of extra set up when instead you could
just use our good friend Homebrew to install it (unfortunately you get the JDK not just the JRE).
\begin{lstlisting}
  brew cask install java
\end{lstlisting}

\subsection{Installing Git}
The Apple managed version of git should have been installed with your developer tools, you can test
this by checking the version.
\begin{lstlisting}
  git --version
\end{lstlisting}

(OPTIONAL) If you want a more recent version, you can install one through brew (or your favorite
package manager). At the time of writing this, the versions only differ by two minor versions, so
the difference is not significant.
\begin{lstlisting}
  brew install git
\end{lstlisting}

\subsection{Installing CMake}
Brew (or otherwise) makes this easy:
\begin{lstlisting}
  brew install cmake
\end{lstlisting}

\subsection{ANTLR 4 C++ Runtime}
This section details how to install the ANTLR 4 C++ runtime on Ubuntu assuming your default shell
is bash. If you've changed your shell from bash it's assumed that you are familiar enough with your
environment that you can modify these steps appropriately.
\begin{enumerate}
  \item
    Choose a directory to download the runtime source to. It's \emph{not necessary} to keep this
    directory around later though it may make your life easier if you need to upgrade. If you don't
    plan on keeping it then \lstinline{$HOME/Downloads} can be a good choice. If you do plan on
    keeping it, then I suggest that you create a singular directory to use as the parent directory
    in all future steps so that it can be cleaned up later easily should you choose to get rid of
    it. A good choice could be \lstinline{$HOME/antlr}. We'll call this directory choice
    \lstinline{SOURCE_PARENT}.

    \lstinline{$HOME} refers to your home directory. \lstinline{echo $HOME} will show you the path.
  \item
    Make your parent directory if necessary. Generally, you want:
    \begin{lstlisting}
      mkdir <SOURCE_PARENT> # E.g. mkdir $HOME/antlr
    \end{lstlisting}
  \item
    Next, we need to clone the source for the runtime from GitHub.
    \begin{lstlisting}
      cd <SOURCE_PARENT>
      git clone git@github.com:antlr/antlr4.git
    \end{lstlisting}
    This should create a new folder called \lstinline{antlr4} in \lstinline{SOURCE_PARENT}. We'll
    refer to this new directory (\lstinline{<SOURCE_PARENT>/antlr4}) as \lstinline{SRC_DIR}.
  \item
    We will be using ANTLR 4.7.1 so we need to change to the git tag for version 4.7.1.
    \begin{lstlisting}
      cd <SRC_DIR>
      git checkout 4.7.1
    \end{lstlisting}
  \item
    Now we need to choose a place to build the runtime. If you've decided to not keep the source
    then the build directory will be useless because it references the source. I would suggest
    creating your build directory inside the source tree (i.e. \lstinline{<SRC_DIR>/antlr4-build})
    so it's easy to clean up. If you're planning on keeping the source then inside your source
    directory is still often a good choice, but in the directory beside the source (i.e.
    \lstinline{<SOURCE_PARENT>/antlr4-build}) is sometimes a good idea too. Again, this can go
    anywhere you choose if you have a personal preference. We'll call this directory choice
    \lstinline{BUILD_PARENT}.
  \item
    We can make the actual build directory:
    \begin{lstlisting}
      cd <BUILD_PARENT>
      mkdir antlr4-build
    \end{lstlisting}
    We'll refer to this new directory (\lstinline{<BUILD_PARENT>/antlr4-build}) as
    \lstinline{BUILD_DIR}.
  \item
    We need to have an install directory prepared before building since it's referenced in the
    build step. This directory will have the headers and compiled ANTLR libraries put into it so
    you cannot delete it. Again, a folder in \lstinline{$HOME} can be a good idea but any
    directory you choose for third party libraries is acceptable. We'll call this directory choice
    \lstinline{INSTALL_PARENT}.
  \item
    We can make the actual install directory:
    \begin{lstlisting}
      cd <INSTALL_PARENT>
      mkdir antlr4-install
    \end{lstlisting}
    We'll refer to this new directory (\lstinline{<INSTALL_PARENT>/antlr4-install}) as
    \lstinline{INSTALL_DIR}.
  \item
    Finally, we're ready to start the actual build process. Let's begin by generating the CMake
    config for the runtime. We need to do this while inside the build directory. As well, we need
    to tell it that we want a release build and to install it to a certain directory.
    \begin{lstlisting}
      cd <BUILD_DIR>
      cmake <SRC_DIR>/runtime/Cpp/ -DCMAKE_BUILD_TYPE=RELEASE -DCMAKE_INSTALL_PREFIX="<INSTALL_DIR>"
    \end{lstlisting}
    You will be presented with come CMake warnings but they're safe to ignore.
  \item
    We can finally run make to build the library and install it. You can make the process
    significantly faster by running with multiple threads using the \lstinline{-j} option and
    specifying a thread count. Using the option without a count will use all cores. Be careful when
    using unlimited threads, the build has failed in the past. This isn't a big issue because you
    can always just try again with a limited number of threads.
    \begin{lstlisting}
      make install -j<number of threads>
    \end{lstlisting}
  \item
    Now we can add the install to your bash profile. Pick your favorite text editor and open
    \lstinline{~/.bash_profile} and add the following lines to the end, substituting appropriately:
    \begin{lstlisting}
      # C415 ANTLR install
      export ANTLR_INS=<INSTALL_DIR>
    \end{lstlisting}
  \item
    If you want to clean up then you can remove the source and build directories. Just remember,
    you may have to download them and set up again if we update later (though it's unlikely).
    \begin{lstlisting}
      rm -rf <SRC_DIR> <BUILD_DIR>
    \end{lstlisting}
\end{enumerate}

\subsection{Installing CLion}
\begin{enumerate}
  \item
    Use Homebrew to install CLion:
    \begin{lstlisting}
      brew cask install clion
    \end{lstlisting}
  \item
    Open CLion (via spotlight: command+space $\rightarrow$ type \texttt{CLion}).
  \item
    Perform the initial set up of CLion.
    \begin{enumerate}
      \item
        Select \texttt{Do not import settings} and click \texttt{OK}.
      \item
        Scroll to the bottom of the license agreement then hit \texttt{Accept}.
      \item
        Choose if you want to share usage statistics.
      \item
        You should be presented with a prompt for your license. Select \texttt{Activate},
        \texttt{JetBrains Account}, enter your UAlberta email address and JetBrains password.
        Click the \texttt{Activate} button.
      \item
        Pick your favorite UI. Then click \texttt{Next: Toolchains}.
      \item
        CLion bundles a version of CMake with it. If you'd prefer to use the one we've just
        installed change \texttt{Bundled} to \lstinline{/usr/local/bin/cmake}. The info text
        beneath should update with a checkmark and the version of your installed cmake. Click
        \texttt{Next: Default Plugins}.
      \item
        You might consider disabling all but the git plugin, and even then, using it is up to you.
        It can be useful to see the color coded files for differences at a glance or track changes
        in a file. You should consider disabling all of the web development plugins. Disabling
        other tools is up to you as well. Now select \texttt{Next: Feature Plugins}
      \item
        Again, the choices here are yours if you like vim, then maybe the vim plugin is up your
        alley. The markdown plugin can be useful as well. You do not need the TeamCity Integration,
        the Lua integration, nor the Swift integration.
        Select \texttt{Start using CLion}
    \end{enumerate}
\end{enumerate}

\subsection{Installing the ANTLR Plugin for CLion}
ANTLR has a CLion integration that gives syntax highlighting as well as tool for visualising the
parse tree for a grammar rule and an input.
\begin{enumerate}
  \item
    Launch CLion by going to the application launcher (finder) and typing \lstinline{clion}. This
    should launch CLion.
  \item
    Open the settings window \texttt{CLion $\rightarrow$ Preferences...}
  \item
    Select \texttt{Plugins} from the menu on the left.
  \item
    Click \texttt{Browse Repositories...} below the plugin list.
  \item
    In the new window, type \texttt{antlr} into the search bar at the top.
  \item
    From the list select \lstinline{ANTLR v4 grammar plugin}
  \item
    Click \texttt{Install} in the right pane and accept the notice.
  \item
    After the install bar ends click the \texttt{Restart CLion} button that should have replaced
    the \texttt{Install} button.
\end{enumerate}

\subsection{Installing ANTLR Generator}
If you'd like to manually generate a listener or visitor you need to have the ANTLR generator.
Follow these steps into install it:
\begin{enumerate}
  \item
    Make the desitination directory. I would suggest putting this in \lstinline{INSTALL_DIR/bin}
    since the CMake projects will already automatically download a copy there and duplicating
    this seems wasteful.
    \begin{lstlisting}
      mkdir <INSTALL_DIR>/bin
      curl http://www.antlr.org/download/antlr-4.7.1-complete.jar > <INSTALL_DIR>/bin/antlr-4.7.1-complete.jar
    \end{lstlisting}
  \item
    Now we can make it easy to use. Add the following lines to your \lstinline{~/.bash_profile}:
    \begin{lstlisting}
      # C415 Antlr Generator
      export CLASSPATH="<INSTALL_DIR>/bin/antlr-4.7.1-complete.jar:$CLASSPATH"
      alias antlr4="java -Xmx500M org.antlr.v4.Tool"
      alias grun='java org.antlr.v4.gui.TestRig'
    \end{lstlisting}
    Restart your terminal for things to take effect. Now these commands should produce useful help
    outputs:
    \begin{lstlisting}
      antlr4
      grun
    \end{lstlisting}
\end{enumerate}


  \subsection{Configuring first project}
    \begin{enumerate}
      \item Click \texttt{Create new project}
      \item Select Java Project and then next to \textit{Project SDK} select \texttt{New} $\rightarrow$ \texttt{JDK}
      \item The installation process will give you an interactive interface (similar to finder) to locate the
      file. At the very top of the window select \texttt{"Computer"}, then follow the directory path Macintosh
      \begin{lstlisting}
        HD/Library/Java/JavaVirtualMachines
      \end{lstlisting}
      and select the folder \texttt{jdk1.8.0\_45.jdk} (or whichever is the current version that you are
      installing).
      \item Click \texttt{Next} then click \texttt{Next} again.
      \item Name your project and click \texttt{Finish}
      \item Open Project manager (Default hotkey CMD-1)
      \item In the project menu that you just opened right-click on the \texttt{src} folder $\rightarrow$ New $\rightarrow$ File
      \item name the file
      \begin{lstlisting}
        hello.g4
      \end{lstlisting}
      \item In the new file put the following content
      \begin{lstlisting}
        grammar hello;

        rule: 'hello' id;
        id  : STRING;

        STRING: CHAR+;
        CHAR: [a-z|A-Z];
        WS: ' '+ -> skip;
      \end{lstlisting}
      \item CLTR-click on \texttt{rule} $\rightarrow$ select \texttt{Test Rule rule}
      \item There are three new boxes that open:
      \begin{enumerate}
        \item a small that appears to be to input a file name
        \item a larger one underneath
        \item To the right of that there is a box on top of which is "Parse tree | Profiler"
      \end{enumerate}
      in the second box type
      \begin{lstlisting}
        hello <your name>
      \end{lstlisting}
      in the third box a tree should appear a parse tree
      \item Congratulations you've made your first Antlr4 parser

    \end{enumerate}

\end{document}
