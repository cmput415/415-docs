\documentclass[../setup.tex]{subfiles}

\begin{document}

\subsection{Installing Oracle Java JDK 8}

  \begin{enumerate}
    \item Go to \href{http://www.oracle.com/technetwork/java/javase/downloads/jdk8-downloads-2133151.html} {the
    Oracle download site} and download the version of oracle jdk8 appropriate to your windows system.

    \item run the Oracle Java installer you just downloaded and follow all the steps
  \end{enumerate}

\subsection{Installing Intellij}
  \begin{enumerate}
    \item Go to the \href{https://www.jetbrains.com/idea/download/index.html} {jetbrains} and download
    \textit{Intellij Community} for windows.

    \item Run the installation wizard you just downloaded \item Follow the instructions and install it
    (Recommended use default setting)

    \item Launch IntellijIDEA to finish installation
    \begin{enumerate}
      \item Select I don't have any configurations.
      \item Select \texttt{JetBrains account} and use the account information you just created to
      authenticate. \textbf{You must have already authenticated it first otherwise this won't work.}
      \item Pick whichever UI is prettiest to you. Then hit \texttt{Next: Default Plug-ins}
      \item Leave Java Frameworks and Build tools enabled everything else can be disabled at your discretion.
      Then hit \texttt{Next: Feature Plug-ins}
      \item None of these are necessary so you can hit \texttt{Start using IntellijIDEA}
    \end{enumerate}
  \end{enumerate}


\subsection{Installing antlr v4 plug-in for IntelliJ}

  \begin{enumerate}
    \item Launch IntelliJ by going to the application launcher and typing
    \begin{lstlisting}
      intellij idea
    \end{lstlisting}
    This should launch Intellij IDEA.

    \item In the bottom corner click \texttt{Configure $\rightarrow$ Plugins}. This will open the plugin manager
    \item Click \texttt{Browse Repositories}
    \item In the search bar type
    \begin{lstlisting}
      antlr
    \end{lstlisting}
    \item From the list select
    \begin{lstlisting}
      ANTLR v4 grammar plugin
    \end{lstlisting}
    and hit \texttt{install plugin}. Hit yes when asked for confirmation.
    \item After the install bar ends hit the button \texttt{restart Intellij}
  \end{enumerate}


\subsection{Getting Antlr4 libraries}

  \begin{enumerate}
    \item Follow the steps on this
    \href{https://github.com/antlr/antlr4/blob/master/doc/getting-started.md}{tutorial}
    \item Follow the steps on this
    \href
    {https://github.com/antlr/stringtemplate4/blob/master/doc/java.md}
    {tutorial}
  \end{enumerate}


\subsection{Configuring first project}

  \begin{enumerate}
    \item Click \texttt{Create new project}
    \item Select Java Project and then next to \textit{Project SDK} select \texttt{New} $\rightarrow$ \texttt{JDK}
    \item The location should be
    \begin{lstlisting}
      C:\Programs Files\Java\jdk<Version>\
    \end{lstlisting}
    \item Click \texttt{Next} then click \texttt{Next} again.
    \item Name your project and click \texttt{Finish} again
    \item Open Project manager (Default hotkey Alt-1)
    \item In the project menu that you just opened right-click on the \texttt{src} folder $\rightarrow$ New $\rightarrow$ File
    \item name the file
    \begin{lstlisting}
      hello.g4
    \end{lstlisting}
    \item In the new file put the following content
    \begin{lstlisting}
      grammar hello;

      rule: 'hello' id;
      id  : STRING;

      STRING: CHAR+;
      CHAR: [a-z|A-Z];
      WS: ' '+ -> skip;
    \end{lstlisting}
    \item right click on \texttt{rule} $\rightarrow$ select \texttt{Test Rule rule}
    \item There are three new boxes that open:
    \begin{enumerate}
      \item a small that appears to be to input a file name
      \item a larger one underneath
      \item To the right of that there is a box on top of which is "Parse tree | Profiler"
    \end{enumerate}
    in the second box type
    \begin{lstlisting}
      hello <your name>
    \end{lstlisting}
    in the third box a tree should appear a parse tree
    \item Congratulations you've made your first Antlr4 parser

  \end{enumerate}
\end{document}
