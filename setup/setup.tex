\documentclass{article}

\usepackage{hyperref} % Almost certainly will need
\usepackage{fullpage} % Good for making PDFs as well
\usepackage{listings} % Needed to insert code
\usepackage{lstautogobble}
\usepackage{subfiles} % Used to separate files to subfiles.

\usepackage[T1]{fontenc}
\usepackage{textcomp}
% This makes it so the web pages don't have indents as most of the time
% they are just annoying
\setlength{\parindent}{0pt}
\lstset{
	upquote=true,
	basicstyle = \ttfamily,
	columns=fullflexible
	escapeinside=||,
	autogobble
}
% --- This section allows for tex4ht only control statments
% From http://tex.stackexchange.com/questions/93852/what-is-the-correct-way-to-check-for-latex-pdflatex-and-html-in-the-same-latex
\makeatletter
\edef\texforht{TT\noexpand\fi
  \@ifpackageloaded{tex4ht}
    {\noexpand\iftrue}
    {\noexpand\iffalse}}
\makeatother
% -----------------------------------------------------------------------------

\begin{document}

% Titles
\ifpdf
	\LARGE
	\textbf{\textit{CMPUT 415} Setup Instructions}
	\normalsize
\else\if\texforht
\fi\fi

These instructions are for setting up your development environment.  There are different instructions for
Ubuntu (Should be applicable for most linux systems), Mac, and Windows environments. \textbf{Make sure you complete the
section on building your first antlr parser or else you will not complete the setup}. Use the instruction relevant to
your machine.

% -----------------------------------------------------------------------------


\subsection{Mac OS X}
This section details how to install the ANTLR 4 C++ runtime on Mac OS X assuming your default shell
is bash. If you've changed your shell from bash it's assumed that you are familiar enough with your
environment that you can modify these steps appropriately.
\begin{enumerate}
  \item
    Ensure you have git installed (it should be installed by default on Ubuntu). Check the version
    to ensure it works:
    \begin{lstlisting}
      git --version
    \end{lstlisting}
    If you don't have it then you should be able to install it using this line:
    \begin{lstlisting}
      brew install git
    \end{lstlisting}
  \item
    Choose a directory to download the runtime source to. It's \emph{not necessary} to keep this
    directory around later though it may make your life easier if you need to upgrade. If you don't
    plan on keeping it then \lstinline{\$HOME/Downloads} can be a good choice. If you do plan on
    keeping it, \lstinline{/opt} can be a good choice but it can go anywhere you choose if you have
    a personal preference. We'll call this directory choice \lstinline{SOURCE\_PARENT}.
  \item
    Next, we need to clone the source for the runtime from GitHub.
    \begin{lstlisting}
      cd <SOURCE_PARENT>
      git clone git@github.com:antlr/antlr4.git
    \end{lstlisting}
    This should create a new folder called \lstinline{antlr4} in \lstinline{SOURCE\_PARENT}. We'll
    refer to this new directory (\lstinline{<SOURCE\_PARENT>/antlr4}) as \lstinline{SRC\_DIR}.
  \item
    We will be using ANTLR 4.7.1 so we need to change to the git tag for version 4.7.1.
    \begin{lstlisting}
      cd <SRC_DIR>
      git checkout 4.7.1
    \end{lstlisting}
  \item
    Now we need to choose a place to build the runtime. If you've decided to not keep the source
    then the build directory will be useless because it references the source. I would suggest
    creating your build directory inside the source tree (i.e. \lstinline{<SRC\_DIR>/build}) so
    it's easy to clean up. If you're planning on keeping the source then inside your source
    directory is still often a good choice, but in the directory beside the source (i.e.
    \lstinline{<SOURCE\_PARENT>/build}) is sometimes a good idea too. Again, this can go anywhere
    you choose if you have a personal preference. We'll call this directory choice
    \lstinline{BUILD\_PARENT}.
  \item
    We can make the actual build directory:
    \begin{lstlisting}
      cd <BUILD_PARENT>
      mkdir antlr4-build
    \end{lstlisting}
    We'll refer to this new directory (\lstinline{<BUILD\_PARENT>/antlr4-build}) as
    \lstinline{BUILD\_DIR}.
  \item
    We need to have an install directory prepared before building since it's referenced in the
    build step. This directory will have the headers and compiled ANTLR libraries put into it so
    you cannot delete it. Again, \lstinline{/opt} can be a good idea but any directory you choose
    for third party libs is acceptable. We'll call this directory choice
    \lstinline{INSTALL\_PARENT}.
  \item
    We can make the actual install directory:
    \begin{lstlisting}
      cd <INSTALL_PARENT>
      mkdir antlr4-install
    \end{lstlisting}
    We'll refer to this new directory (\lstinline{<INSTALL\_PARENT>/antlr4-install}) as
    \lstinline{INSTALL\_DIR}.
  \item
    Finally, we're ready to start the actual build process. Let's being by generating the CMake
    config for the runtime. We need to do this while inside the build directory. As well, we need
    to tell it that we want to install it to a certain directory.
    \begin{lstlisting}
      cd <BUILD_DIR>
      cmake <SRC_DIR>/runtime/Cpp/ -DCMAKE_INSTALL_PREFIX="<INSTALL_DIR>"
    \end{lstlisting}
  \item
    We can finally run make to build the library and install it. You can make the process
    significantly faster by running with multiple threads using the \lstinline{-j} option and
    specifying a thread count. Using the option without a count will use all cores.
    \begin{lstlisting}
      make install -j<number of threads>
    \end{lstlisting}
  \item
    Now we can add the install to your bashrc. Pick your favorite text editor and open
    \lstinline{~/.bashrc} and add the following lines to the end, substituting appropriately:
    \begin{lstlisting}
      # C415 ANTLR install
      export ANTLR_INS=<INSTALL_DIR>
    \end{lstlisting}
  \item
    If you want to clean up then you can remove the source and build directories. Just remember,
    you may have to download them and set up again if we update later (though it's unlikely).
    \begin{lstlisting}
      rm -rf <SRC_DIR> <BUILD_DIR>
    \end{lstlisting}
\end{enumerate}

\section{First Steps}
\subsection{JetBrains License}
You need a valid JetBrains account to use CLion. Luckily, you're a student and that gets you
free things. If you already have an account with a student license, you can skip this.
\begin{enumerate}
  \item
    Fill out \href{https://www.jetbrains.com/shop/eform/students} {this form} to start getting
    your student license. Select \texttt{I'm a student}, enter your first and last name, enter
    your \textbf{UAlberta email address}, select \texttt{Canada}, and finally agree to the
    account agreement.
  \item
    Go to your email inbox, find the email with the subject \texttt{JetBrains Educational Pack
    Confirmation}. Open it and click the link \texttt{Confirm Request}.
  \item
    You should see a new page with the header \texttt{Congrats! You've been approved!}.
  \item
    There should be a new email in your inbox with the subject \texttt{JetBrains Student License
    Confirmation}. Open it and click the link \texttt{Activate Educational License}.
  \item
    Fill out the form to create your account. Enter your first name, last name, and username.
    Choose an appropriate password and accept the account agreement. It's your choice to
    consent to the use of your data, but it's not necessary.
  \item
    Your account should be usable with CLion now.
\end{enumerate}

\section{Ubuntu}
This section details how to setup the Ubuntu development environment.
\subfile{envs/ubuntu.tex}

\section{Windows}
This section details how to setup the Windows development environment.
\subfile{envs/windows.tex}

\section{Mac OS X}
This section details how to setup the Mac OS X development environment.
\subfile{envs/macos.tex}

\section{CS Computers}
This section details how to setup your environment to use the resources already present on the CSC
lab computers.
\subfile{envs/csc.tex}

\end{document}
