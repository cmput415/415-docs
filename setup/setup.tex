\documentclass{article}

\usepackage{hyperref} % Almost certainly will need
\usepackage{fullpage} % Good for making PDFs as well
\usepackage{listings} % Needed to insert code
\usepackage{lstautogobble}

\usepackage[T1]{fontenc}
\usepackage{textcomp}
% This makes it so the web pages don't have indents as most of the time
% they are just annoying
\setlength{\parindent}{0pt}
\lstset{
	upquote=true,
	basicstyle = \ttfamily,
	columns=fullflexible
	escapeinside=||,
	autogobble
}
% --- This section allows for tex4ht only control statments
% From http://tex.stackexchange.com/questions/93852/what-is-the-correct-way-to-check-for-latex-pdflatex-and-html-in-the-same-latex
\makeatletter
\edef\texforht{TT\noexpand\fi
  \@ifpackageloaded{tex4ht}
    {\noexpand\iftrue}
    {\noexpand\iffalse}}
\makeatother
% -----------------------------------------------------------------------------

\begin{document}

% Titles
\ifpdf
	\LARGE
	\textbf{\textit{CMPUT 415} Setup Instructions}
	\normalsize
\else\if\texforht
\fi\fi

These instructions are for setting up your development environment.  There are different instructions for
Ubuntu (Should be applicable for most linux systems), Mac, and Windows environments. \textbf{Make sure you complete the
section on building your first antlr parser or else you will not complete the setup}. Use the instruction relevant to
your machine.

% -----------------------------------------------------------------------------


\subsection{Mac OS X}
This section details how to install the ANTLR 4 C++ runtime on Mac OS X assuming your default shell
is bash. If you've changed your shell from bash it's assumed that you are familiar enough with your
environment that you can modify these steps appropriately.
\begin{enumerate}
  \item
    Ensure you have git installed (it should be installed by default on Ubuntu). Check the version
    to ensure it works:
    \begin{lstlisting}
      git --version
    \end{lstlisting}
    If you don't have it then you should be able to install it using this line:
    \begin{lstlisting}
      brew install git
    \end{lstlisting}
  \item
    Choose a directory to download the runtime source to. It's \emph{not necessary} to keep this
    directory around later though it may make your life easier if you need to upgrade. If you don't
    plan on keeping it then \lstinline{\$HOME/Downloads} can be a good choice. If you do plan on
    keeping it, \lstinline{/opt} can be a good choice but it can go anywhere you choose if you have
    a personal preference. We'll call this directory choice \lstinline{SOURCE\_PARENT}.
  \item
    Next, we need to clone the source for the runtime from GitHub.
    \begin{lstlisting}
      cd <SOURCE_PARENT>
      git clone git@github.com:antlr/antlr4.git
    \end{lstlisting}
    This should create a new folder called \lstinline{antlr4} in \lstinline{SOURCE\_PARENT}. We'll
    refer to this new directory (\lstinline{<SOURCE\_PARENT>/antlr4}) as \lstinline{SRC\_DIR}.
  \item
    We will be using ANTLR 4.7.1 so we need to change to the git tag for version 4.7.1.
    \begin{lstlisting}
      cd <SRC_DIR>
      git checkout 4.7.1
    \end{lstlisting}
  \item
    Now we need to choose a place to build the runtime. If you've decided to not keep the source
    then the build directory will be useless because it references the source. I would suggest
    creating your build directory inside the source tree (i.e. \lstinline{<SRC\_DIR>/build}) so
    it's easy to clean up. If you're planning on keeping the source then inside your source
    directory is still often a good choice, but in the directory beside the source (i.e.
    \lstinline{<SOURCE\_PARENT>/build}) is sometimes a good idea too. Again, this can go anywhere
    you choose if you have a personal preference. We'll call this directory choice
    \lstinline{BUILD\_PARENT}.
  \item
    We can make the actual build directory:
    \begin{lstlisting}
      cd <BUILD_PARENT>
      mkdir antlr4-build
    \end{lstlisting}
    We'll refer to this new directory (\lstinline{<BUILD\_PARENT>/antlr4-build}) as
    \lstinline{BUILD\_DIR}.
  \item
    We need to have an install directory prepared before building since it's referenced in the
    build step. This directory will have the headers and compiled ANTLR libraries put into it so
    you cannot delete it. Again, \lstinline{/opt} can be a good idea but any directory you choose
    for third party libs is acceptable. We'll call this directory choice
    \lstinline{INSTALL\_PARENT}.
  \item
    We can make the actual install directory:
    \begin{lstlisting}
      cd <INSTALL_PARENT>
      mkdir antlr4-install
    \end{lstlisting}
    We'll refer to this new directory (\lstinline{<INSTALL\_PARENT>/antlr4-install}) as
    \lstinline{INSTALL\_DIR}.
  \item
    Finally, we're ready to start the actual build process. Let's being by generating the CMake
    config for the runtime. We need to do this while inside the build directory. As well, we need
    to tell it that we want to install it to a certain directory.
    \begin{lstlisting}
      cd <BUILD_DIR>
      cmake <SRC_DIR>/runtime/Cpp/ -DCMAKE_INSTALL_PREFIX="<INSTALL_DIR>"
    \end{lstlisting}
  \item
    We can finally run make to build the library and install it. You can make the process
    significantly faster by running with multiple threads using the \lstinline{-j} option and
    specifying a thread count. Using the option without a count will use all cores.
    \begin{lstlisting}
      make install -j<number of threads>
    \end{lstlisting}
  \item
    Now we can add the install to your bashrc. Pick your favorite text editor and open
    \lstinline{~/.bashrc} and add the following lines to the end, substituting appropriately:
    \begin{lstlisting}
      # C415 ANTLR install
      export ANTLR_INS=<INSTALL_DIR>
    \end{lstlisting}
  \item
    If you want to clean up then you can remove the source and build directories. Just remember,
    you may have to download them and set up again if we update later (though it's unlikely).
    \begin{lstlisting}
      rm -rf <SRC_DIR> <BUILD_DIR>
    \end{lstlisting}
\end{enumerate}

\section{First Steps}
\subsection{JetBrains License}
You need a valid JetBrains account to use CLion. Luckily, you're a student and that gets you
free things. If you already have an account with a student license, you can skip this.
\begin{enumerate}
  \item
    Fill out \href{https://www.jetbrains.com/shop/eform/students} {this form} to start getting
    your student license. Select \texttt{I'm a student}, enter your first and last name, enter
    your \textbf{UAlberta email address}, select \texttt{Canada}, and finally agree to the
    account agreement.
  \item
    Go to your email inbox, find the email with the subject \texttt{JetBrains Educational Pack
    Confirmation}. Open it and click the link \texttt{Confirm Request}.
  \item
    You should see a new page with the header \texttt{Congrats! You've been approved!}.
  \item
    There should be a new email in your inbox with the subject \texttt{JetBrains Student License
    Confirmation}. Open it and click the link \texttt{Activate Educational License}.
  \item
    Fill out the form to create your account. Enter your first name, last name, and username.
    Choose an appropriate password and accept the account agreement. It's your choice to
    consent to the use of your data, but it's not necessary.
  \item
    Your account should be usable with CLion now.
\end{enumerate}

\section{Ubuntu}
This section details how to setup the Ubuntu development environment.

\subsection{Installing OpenJDK JRE 8}
The Java runtime environment (JRE) is required to run the ANTLR generator. OpenJDK's JRE is easier
to install than Oracle's, so we'll use that.

\begin{lstlisting}
	sudo apt-get update
	sudo apt-get install openjdk-8-jre
\end{lstlisting}

\subsection{Installing Git}
Git should be installed by default in Ubuntu. If you've removed it or it is otherwise unavailable
then you can install it using this command:
\begin{lstlisting}
  sudo apt-get update
  sudo apt-get install git
\end{lstlisting}

\subsection{ANTLR 4 C++ Runtime}
This section details how to install the ANTLR 4 C++ runtime on Ubuntu assuming your default shell
is bash. If you've changed your shell from bash it's assumed that you are familiar enough with your
environment that you can modify these steps appropriately.

\begin{enumerate}
  \item
    Choose a directory to download the runtime source to. It's \emph{not necessary} to keep this
    directory around later though it may make your life easier if you need to upgrade. If you don't
    plan on keeping it then \lstinline{\$HOME/Downloads} can be a good choice. If you do plan on
    keeping it, \lstinline{/opt} can be a good choice but it can go anywhere you choose if you have
    a personal preference. You can read about how to use \lstinline{/opt}
    \href{https://askubuntu.com/a/34922/550300} {here}. We'll call this directory choice
    \lstinline{SOURCE\_PARENT}.
  \item
    Next, we need to clone the source for the runtime from GitHub.
    \begin{lstlisting}
      cd <SOURCE_PARENT>
      git clone git@github.com:antlr/antlr4.git
    \end{lstlisting}
    This should create a new folder called \lstinline{antlr4} in \lstinline{SOURCE\_PARENT}. We'll
    refer to this new directory (\lstinline{<SOURCE\_PARENT>/antlr4}) as \lstinline{SRC\_DIR}.
  \item
    We will be using ANTLR 4.7.1 so we need to change to the git tag for version 4.7.1.
    \begin{lstlisting}
      cd <SRC_DIR>
      git checkout 4.7.1
    \end{lstlisting}
  \item
    Now we need to choose a place to build the runtime. If you've decided to not keep the source
    then the build directory will be useless because it references the source. I would suggest
    creating your build directory inside the source tree (i.e. \lstinline{<SRC\_DIR>/build}) so
    it's easy to clean up. If you're planning on keeping the source then inside your source
    directory is still often a good choice, but in the directory beside the source (i.e.
    \lstinline{<SOURCE\_PARENT>/build}) is sometimes a good idea too. Again, this can go anywhere
    you choose if you have a personal preference. We'll call this directory choice
    \lstinline{BUILD\_PARENT}.
  \item
    We can make the actual build directory:
    \begin{lstlisting}
      cd <BUILD_PARENT>
      mkdir antlr4-build
    \end{lstlisting}
    We'll refer to this new directory (\lstinline{<BUILD\_PARENT>/antlr4-build}) as
    \lstinline{BUILD\_DIR}.
  \item
    We need to have an install directory prepared before building since it's referenced in the
    build step. This directory will have the headers and compiled ANTLR libraries put into it so
    you cannot delete it. Again, \lstinline{/opt} can be a good idea but any directory you choose
    for third party libs is acceptable. We'll call this directory choice
    \lstinline{INSTALL\_PARENT}.
  \item
    We can make the actual install directory:
    \begin{lstlisting}
      cd <INSTALL_PARENT>
      mkdir antlr4-install
    \end{lstlisting}
    We'll refer to this new directory (\lstinline{<INSTALL\_PARENT>/antlr4-install}) as
    \lstinline{INSTALL\_DIR}.
  \item
    Finally, we're ready to start the actual build process. Let's being by generating the CMake
    config for the runtime. We need to do this while inside the build directory. As well, we need
    to tell it that we want to install it to a certain directory.
    \begin{lstlisting}
      cd <BUILD_DIR>
      cmake <SRC_DIR>/runtime/Cpp/ -DCMAKE_INSTALL_PREFIX="<INSTALL_DIR>"
    \end{lstlisting}
  \item
    We can finally run make to build the library and install it. You can make the process
    significantly faster by running with multiple threads using the \lstinline{-j} option and
    specifying a thread count. Using the option without a count will use all cores.
    \begin{lstlisting}
      make install -j<number of threads>
    \end{lstlisting}
  \item
    Now we can add the install to your bashrc. Pick your favorite text editor and open
    \lstinline{~/.bashrc} and add the following lines to the end, substituting appropriately:
    \begin{lstlisting}
      # C415 ANTLR install
      export ANTLR_INS=<INSTALL_DIR>
    \end{lstlisting}
  \item
    If you want to clean up then you can remove the source and build directories. Just remember,
    you may have to download them and set up again if we update later (though it's unlikely).
    \begin{lstlisting}
      rm -rf <SRC_DIR> <BUILD_DIR>
    \end{lstlisting}
\end{enumerate}

\subsection{Installing CLion}
\begin{enumerate}
	\item
    Go to the \href{https://www.jetbrains.com/clion/download/\#section=linux} {download page} and
    download \textit{CLion} for Linux.
	\item
    Assuming you've downloaded the tarball to your \lstinline{~/Downloads} folder, you can extract
    it to \lstinline{/opt/} using the following command:
  	\begin{lstlisting}
  		tar -xzf CLion-<version>.tar.gz -C /opt/
  	\end{lstlisting}
  	\textit{(If you are confident about your ability to setup your own install you can put it
    elsewhere but you will be on your own.)}
	\item
    Execute the installer:
  	\begin{lstlisting}
  		/opt/CLion-<version>/bin/CLion.sh
  	\end{lstlisting}
	\item
    Follow the installation wizard.
  	\begin{enumerate}
  		\item
        Select ``Do not import settings'' and hit \texttt{OK}.
  		\item
        Pick your favorite UI. Then hit \texttt{Next: Default Plugins}
  		\item
        You might consider disabling all but the git plugin, and even then, using it is up to you.
        It can be useful to see the color coded files for differences at a glance or track changes
        in a file. You should consider disabling all of the web development plugins. Disabling
        other tools is up to you as well. Now select \texttt{Next: Feature Plugins}
  		\item
        Again, the choices here are yours if you like vim, then maybe the vim plugin is up your
        alley. The markdown plugin can be useful as well. You do not need the TeamCity Integration.
        Select \texttt{Start using CLion}
  	\end{enumerate}
\end{enumerate}
	\subsection{Installing antlr v4 plug-in for Intellij}

		\begin{enumerate}
			\item Launch Intellij by going to the application launcher and typing
			\begin{lstlisting}
				intellij idea
			\end{lstlisting}
			This should launch Intellij IDEA.

			\item In the bottom corner click \texttt{Configure $\rightarrow$ Plugins}. This will open the plugin manager

			\item Click \texttt{Browse Repositories}

			\item In the search bar type
			\begin{lstlisting}
				antlr
			\end{lstlisting}

			\item From the list select
			\begin{lstlisting}
				ANTLR v4 grammar plugin
			\end{lstlisting}
			and hit \texttt{install plugin}. Hit yes when asked for confirmation.

			\item After the install bar ends hit the button \texttt{restart Intellij}
		\end{enumerate}


	\subsection{Getting Antlr4 libraries}
		\begin{enumerate}
			\item Follow the steps on this
			\href{https://github.com/antlr/antlr4/blob/master/doc/getting-started.md}{tutorial}
			\item Follow the steps on this
			\href
			{https://github.com/antlr/stringtemplate4/blob/master/doc/java.md}
			{tutorial}
		\end{enumerate}



	\subsection{Configuring first project}
		\begin{enumerate}
			\item Click \texttt{Create new project}

			\item Select Java Project and then next to \textit{Project SDK} select \texttt{New} $\rightarrow$ \texttt{JDK}

			\item The location should be
			\begin{lstlisting}
				/usr/lib/jvm/java-8-oracle/
			\end{lstlisting}

			\item Click \texttt{Next} then click \texttt{Next} again.

			\item Name your project and click \texttt{Finish}

			\item Open Project manager (Default hotkey Alt-1)

			\item In the project menu that you just opened right-click on the \texttt{src} folder $\rightarrow$ New $\rightarrow$ File

			\item name the file
			\begin{lstlisting}
				hello.g4
			\end{lstlisting}

			\item In the new file put the following content
			\begin{lstlisting}
				grammar hello;

				rule: 'hello' id;
				id  : STRING;

				STRING: CHAR+;
				CHAR: [a-z|A-Z];
				WS: ' '+ -> skip;
			\end{lstlisting}

			\item right click on \texttt{rule} $\rightarrow$ select \texttt{Test Rule rule}

			\item There are three new boxes that open:
			\begin{enumerate}
				\item a small that appears to be to input a file name
				\item a larger one underneath
				\item To the right of that there is a box on top of which is "Parse tree | Profiler"
			\end{enumerate}
			in the second box type
			\begin{lstlisting}
				hello <your name>
			\end{lstlisting}
			in the third box a tree should appear a parse tree

			\item Congratulations you've made your first Antlr4 parser

		\end{enumerate}



\section{Windows}

	\subsection{Installing Oracle Java JDK 8}

		\begin{enumerate}
			\item Go to \href{http://www.oracle.com/technetwork/java/javase/downloads/jdk8-downloads-2133151.html} {the
			Oracle download site} and download the version of oracle jdk8 appropriate to your windows system.

			\item run the Oracle Java installer you just downloaded and follow all the steps
		\end{enumerate}

	\subsection{Installing Intellij}
		\begin{enumerate}
			\item Go to the \href{https://www.jetbrains.com/idea/download/index.html} {jetbrains} and download
			\textit{Intellij Community} for windows.

			\item Run the installation wizard you just downloaded \item Follow the instructions and install it
			(Recommended use default setting)

			\item Launch IntellijIDEA to finish installation
			\begin{enumerate}
				\item Select I don't have any configurations.
				\item Select \texttt{JetBrains account} and use the account information you just created to
				authenticate. \textbf{You must have already authenticated it first otherwise this won't work.}
				\item Pick whichever UI is prettiest to you. Then hit \texttt{Next: Default Plug-ins}
				\item Leave Java Frameworks and Build tools enabled everything else can be disabled at your discretion.
				Then hit \texttt{Next: Feature Plug-ins}
				\item None of these are necessary so you can hit \texttt{Start using IntellijIDEA}
			\end{enumerate}
		\end{enumerate}


	\subsection{Installing antlr v4 plug-in for IntelliJ}

		\begin{enumerate}
			\item Launch IntelliJ by going to the application launcher and typing
			\begin{lstlisting}
				intellij idea
			\end{lstlisting}
			This should launch Intellij IDEA.

			\item In the bottom corner click \texttt{Configure $\rightarrow$ Plugins}. This will open the plugin manager
			\item Click \texttt{Browse Repositories}
			\item In the search bar type
			\begin{lstlisting}
				antlr
			\end{lstlisting}
			\item From the list select
			\begin{lstlisting}
				ANTLR v4 grammar plugin
			\end{lstlisting}
			and hit \texttt{install plugin}. Hit yes when asked for confirmation.
			\item After the install bar ends hit the button \texttt{restart Intellij}
		\end{enumerate}


	\subsection{Getting Antlr4 libraries}

		\begin{enumerate}
			\item Follow the steps on this
			\href{https://github.com/antlr/antlr4/blob/master/doc/getting-started.md}{tutorial}
			\item Follow the steps on this
			\href
			{https://github.com/antlr/stringtemplate4/blob/master/doc/java.md}
			{tutorial}
		\end{enumerate}


	\subsection{Configuring first project}

		\begin{enumerate}
			\item Click \texttt{Create new project}
			\item Select Java Project and then next to \textit{Project SDK} select \texttt{New} $\rightarrow$ \texttt{JDK}
			\item The location should be
			\begin{lstlisting}
				C:\Programs Files\Java\jdk<Version>\
			\end{lstlisting}
			\item Click \texttt{Next} then click \texttt{Next} again.
			\item Name your project and click \texttt{Finish} again
			\item Open Project manager (Default hotkey Alt-1)
			\item In the project menu that you just opened right-click on the \texttt{src} folder $\rightarrow$ New $\rightarrow$ File
			\item name the file
			\begin{lstlisting}
				hello.g4
			\end{lstlisting}
			\item In the new file put the following content
			\begin{lstlisting}
				grammar hello;

				rule: 'hello' id;
				id  : STRING;

				STRING: CHAR+;
				CHAR: [a-z|A-Z];
				WS: ' '+ -> skip;
			\end{lstlisting}
			\item right click on \texttt{rule} $\rightarrow$ select \texttt{Test Rule rule}
			\item There are three new boxes that open:
			\begin{enumerate}
				\item a small that appears to be to input a file name
				\item a larger one underneath
				\item To the right of that there is a box on top of which is "Parse tree | Profiler"
			\end{enumerate}
			in the second box type
			\begin{lstlisting}
				hello <your name>
			\end{lstlisting}
			in the third box a tree should appear a parse tree
			\item Congratulations you've made your first Antlr4 parser

		\end{enumerate}

\section{Mac OS X}

\subsection{Installing Oracle Java JRE 8}
\begin{enumerate}
  \item
    Go to
    \href{http://www.oracle.com/technetwork/java/javase/downloads/jre8-downloads-2133155.html} {the
    Oracle download page} and download the latest Mac OS X .dmg.
  \item
    Run the Oracle Java installer follow all the steps.
\end{enumerate}

\subsection{Installing Homebrew}
Homebrew is a package manager for Mac OS X. If you don't have it yet, you can read about it
\href{https://brew.sh/} {here}. Otherwise, install it via the command on their front page:
\begin{lstlisting}
  /usr/bin/ruby -e "$(curl -fsSL https://raw.githubusercontent.com/Homebrew/install/master/install)"
\end{lstlisting}

\subsection{Installing Git}
Git should be installed by default in Ubuntu. If you've removed it or it is otherwise unavailable
then you can install it using this command:

\begin{lstlisting}
  brew install git
\end{lstlisting}


	\subsection{Installing Intellij}

		\begin{enumerate}
			\item Go to the \href{https://www.jetbrains.com/idea/download/index.html} {jetbrains} and download
			\textit{Intellij Community} for OSx.
			\item Launch the installer by double clicking on the \textit{.dmg}
			\item Open your \texttt{Applications} and launch Intellij IDEA 14.app to finish the installation
			\begin{enumerate}
				\item Select I don't have any configurations.
				\item Select \texttt{JetBrains account} and use the account information you just created to
				authenticate. \textbf{You must have already authenticated it first otherwise this won't work.}
				\item Pick whichever UI is prettiest to you. Then hit \texttt{Next: Keymaps}
				\item Select which keymap scheme you want. Then hit \texttt{Next: Default Plug-ins}
				\item Leave Java Frameworks and Build tools enabled everything else can be disabled at your discretion.
				Then hit \texttt{Next: Feature Plug-ins}
				\item None of these are necessary so you can hit \texttt{Start using IntellijIDEA}
			\end{enumerate}
		\end{enumerate}

	\subsection{Installing antlr v4 plug-in for Intellij}
		\begin{enumerate}
			\item Launch Intellij
			\item In the bottom corner click \texttt{Configure $\rightarrow$ Plugins}. This will open the plugin manager
			\item Click \texttt{Browse Repositories}
			\item In the search bar type
			\begin{lstlisting}
				antlr
			\end{lstlisting}
			\item From the list select
			\begin{lstlisting}
				ANTLR v4 grammar plugin
			\end{lstlisting}
			and hit \texttt{install plugin}. Hit yes when asked for confirmation.
			\item After the install bar ends hit the button \texttt{restart Intellij}
		\end{enumerate}


	\subsection{Getting Antlr4 libraries}
		\begin{enumerate}
			\item Follow the steps on this
			\href{https://github.com/antlr/antlr4/blob/master/doc/getting-started.md}{tutorial}
			\item Follow the steps on this
			\href
			{https://github.com/antlr/stringtemplate4/blob/master/doc/java.md}
			{tutorial}
		\end{enumerate}


	\subsection{Configuring first project}
		\begin{enumerate}
			\item Click \texttt{Create new project}
			\item Select Java Project and then next to \textit{Project SDK} select \texttt{New} $\rightarrow$ \texttt{JDK}
			\item The installation process will give you an interactive interface (similar to finder) to locate the
			file. At the very top of the window select \texttt{"Computer"}, then follow the directory path Macintosh
			\begin{lstlisting}
				HD/Library/Java/JavaVirtualMachines
			\end{lstlisting}
			and select the folder \texttt{jdk1.8.0\_45.jdk} (or whichever is the current version that you are
			installing).
			\item Click \texttt{Next} then click \texttt{Next} again.
			\item Name your project and click \texttt{Finish}
			\item Open Project manager (Default hotkey CMD-1)
			\item In the project menu that you just opened right-click on the \texttt{src} folder $\rightarrow$ New $\rightarrow$ File
			\item name the file
			\begin{lstlisting}
				hello.g4
			\end{lstlisting}
			\item In the new file put the following content
			\begin{lstlisting}
				grammar hello;

				rule: 'hello' id;
				id  : STRING;

				STRING: CHAR+;
				CHAR: [a-z|A-Z];
				WS: ' '+ -> skip;
			\end{lstlisting}
			\item CLTR-click on \texttt{rule} $\rightarrow$ select \texttt{Test Rule rule}
			\item There are three new boxes that open:
			\begin{enumerate}
				\item a small that appears to be to input a file name
				\item a larger one underneath
				\item To the right of that there is a box on top of which is "Parse tree | Profiler"
			\end{enumerate}
			in the second box type
			\begin{lstlisting}
				hello <your name>
			\end{lstlisting}
			in the third box a tree should appear a parse tree
			\item Congratulations you've made your first Antlr4 parser

		\end{enumerate}

\section{CS Computers}



	\subsection{Environment variables}
		If you're working on the computers on the CS network then most of the installation is already done for you. However, some steps are needed for the final setup to work properly. To simplify
		the execution of these steps we have created a script to do most of the work. Download the script \href{static/setup_antlr.sh}{here} and
		then run it.

	\subsection{Configuring first project}
		\begin{enumerate}
			\item Click \texttt{Create new project}

			\item Select Java Project and then next to \textit{Project SDK} select \texttt{New} $\rightarrow$ \texttt{JDK}

			\item The location should be
			\begin{lstlisting}
				/usr/lib/jvm/java-8-oracle/
			\end{lstlisting}

			\item Click \texttt{Next} then click \texttt{Next} again.

			\item Name your project and click \texttt{Finish}

			\item Open Project manager (Default hotkey Alt-1)

			\item In the project menu that you just opened right-click on the \texttt{src} folder $\rightarrow$ New $\rightarrow$ File

			\item name the file
			\begin{lstlisting}
				hello.g4
			\end{lstlisting}

			\item In the new file put the following content
			\begin{lstlisting}
				grammar hello;

				rule: 'hello' id;
				id  : STRING;

				STRING: CHAR+;
				CHAR: [a-z|A-Z];
				WS: ' '+ -> skip;
			\end{lstlisting}

			\item right click on \texttt{rule} $\rightarrow$ select \texttt{Test Rule rule}

			\item There are three new boxes that open:
			\begin{enumerate}
				\item a small that appears to be to input a file name
				\item a larger one underneath
				\item To the right of that there is a box on top of which is "Parse tree | Profiler"
			\end{enumerate}
			in the second box type
			\begin{lstlisting}
				hello <your name>
			\end{lstlisting}
			in the third box a tree should appear a parse tree

			\item Congratulations you've made your first Antlr4 parser

		\end{enumerate}

\end{document}
